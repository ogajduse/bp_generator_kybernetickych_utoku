\chapter{Dokumentace nástroje DoSgen}
%makefiles zminit

\section{Instalace}
V této kapitole bude popsána instalace nástroje DoSgen na operačním systému Fedora. Jsou zde zmíněny knihovny, jež je třeba instalovat pro jeho plnou funkčnost. 

\subsection{Instalace závislostí}
\noindent Instalace následujících knihoven je nutností, jelikož jsou přímými závislostmi nástroje trafgen. Výjimkou je knihovna \texttt{libssh}, na které je závislý slovníkový útok.
\begin{lstlisting}[language=bash]
dnf -y install flex bison libnl3-devel libssh-devel
\end{lstlisting}

\noindent Dále je třeba nainstalovat nástroje potřebné ke kompilaci kódu.
\begin{lstlisting}[language=bash]
dnf -y install make pkg-config gcc-c++
\end{lstlisting}

\noindent Chceme-li sestavit dokumentaci k nástroji DoSgen, je nutné mít nainstalován balíček \texttt{rubygem-ronn}, jež zajišťuje konverzi manuálové stránky z jednoduše formátovaného souboru v jazyce Markdown do formátu \texttt{roff}, který je používán pro psaní manuálových stránek.

\begin{lstlisting}[language=bash]
dnf -y install rubygem-ronn
\end{lstlisting}

\noindent Webové rozhraní je závislé na \texttt{nodejs} a \texttt{npm} balíčcích.
\begin{lstlisting}[language=bash]
dnf -y install nodejs npm
\end{lstlisting}



\noindent Instalace následujících balíčků je pouze volitelné, jde o užitečné nástroje, které najdou své místo vedle nástroje DoSgen.
\begin{lstlisting}[language=bash]
dnf -y install tcptrack nmap httping john iftop nload \
ntp net-snmp-utils iputils net-tools
\end{lstlisting}

\subsection{Kompilace}
\noindent Dále je potřeba přeložit zdrojový kód. V rámci této práce byly vyvinuty soubory \texttt{Makefile} v každém podadresáři kořenového adresáře. V kořenovém adresáři se nachází hlavní \texttt{Makefile}, který spouští všechny Makefile obsažené v podadresářích. Tyto soubory obsahují příkazy pro kompilaci a instalaci kódu a zajišťují správné pořadí jejich vykonání.

\begin{lstlisting}[language=bash]
make
\end{lstlisting}

V rámci této práce byl vytvořen také Dockerfile, pomocí kterého lze sestavit kontejner s nainstalovaným nástrojem DoSgen. Výhoda tohoto kontejneru spočívá v tom, že procesy spuštěné v kontejneru jsou oddělené od procesů v hostitelské stanici. Tento kontejner staví na operačním systému Fedora ve verzi 27. Sestavení kontejneru a jeho spuštění lze docílit následovně.
\begin{lstlisting}
docker build -t <NÁZEV_OBRAZU_KONTEJNERU> .
docker run <NÁZEV_OBRAZU_KONTEJNERU>
\end{lstlisting}

\subsection{Webové rozhraní}
Pro spuštění webového rozhraní je nutno nainstalovat Node.js balíčky.
\begin{lstlisting}[language=bash]
cd dosgen-web
make install-npm
# nebo
npm install
\end{lstlisting}

Chceme-li spouštět webovou aplikaci jako démona běžícího na pozadí, je nutno nainstalovat inicializační skript \texttt{dosgen-web.init} do adresáře \texttt{/etc/init.d} a enviromentální soubor do \texttt{/etc/sysconfig}. Používáme-li systém se systemd, pak namísto inicializačního skriptu instalujeme jednotku \texttt{dosgen-web.service} do složky \texttt{/usr/lib/systemd/system}.

Tyto kroky není nutné provádět ručně. Pro tento účel je v Makefile patřičný cíl, který provede instalaci těchto souborů.
\begin{lstlisting}[language=bash]
make install
\end{lstlisting}

V souboru s proměnnými prostředí a souboru \texttt{dosgen.js} je nutno nastavit cestu k nástroji DoSgen a jeho binárnímu souboru. V těchto souborech jsou přítomny komentáře, které uživatele navedou.

\subsection{Dokumentace}
Chceme-li si zobrazit manuálovou stránku pro nástroj DoSgen, je třeba provést překlad.
\begin{lstlisting}[language=bash]
cd man
make
\end{lstlisting}

\subsection{RPM balíček}
Kořenový adresář projektu obsahuje soubor \texttt{dosgen.spec}, jež obsahuje kompletní instrukce pro sestavení RPM balíčku pro Linuxové distribuce založené na balíčkovacím systému RPM. Specfile používá příkaz \texttt{make} ke kompilaci zdrojového kódu i jeho následnou instalaci. Pro sestavení balíčku je nutno vytvořit SRPM balíček a poté spustit proces sestavení samotného RPM balíčku. Pro vytvoření výsledného balíčku je možno použít software \texttt{mock}. 

\begin{lstlisting}[language=bash]
make test-srpm
mock -r fedora-rawhide-x86_64 \
rpmbuild/SRPMS/dosgen-*.src.rpm
\end{lstlisting}

\section{Webové rozhraní}
Webové rozhraní lze spustit jako službu na pozadí,

\begin{lstlisting}[language=bash]
service dosgen-web start
\end{lstlisting}

\noindent nebo lze jej spustit přímo z adresáře \texttt{dosgen-web} spuštěním inicializačního skriptu.

\begin{lstlisting}[language=bash]
./dosgen-web.init start
\end{lstlisting}
%todo přidat obrázek webui
Webové rozhraní je dostupné na adrese:
\begin{lstlisting}
https://<IP adresa stanice>:8888
\end{lstlisting}

Po načtení webového rozhraní je nutno přihlásit se údaji zadanými v souboru \texttt{config.json}. Na hlavní stránce v panelu \texttt{Common Settings} je nutno zadat odchozí síťové rozhraní. Čas útoku a počet procesů jsou volitelné. V nabídce \texttt{Add your attack} si lze vybrat požadovaný útok a následně je uživatel vyzván k zadání parametrů pro tento útok v dalším panelu.

Tlačítka \texttt{Run!} a \texttt{Stop!} slouží ke spuštění a zastavení konkrétního útoku.

%TODO ?? \section{Konfigurační jazyk nástroje Trafgen}





\chapter{Obsah přiloženého CD}
% TODO: vlozit pdf
{\small
%
\dirtree{%.
.1 /\DTcomment{kořenový adresář přiloženého CD}.
.2 configure.
.2 Dockerfile.
.2 dosgen.
.3 argsParse.c.
.3 argsParse.cpp.
.3 argsParse.h.
.3 dosgen.c.
.3 httpFlood.cpp.
.3 httpFlood.h.
.3 libdos.h.
.3 Makefile.
.3 packet.
.3 rudy.cpp.
.3 rudy.h.
.3 slowloris.cpp.
.3 slowloris.h.
.3 slowRead.cpp.
.3 slowRead.h.
.3 sshPass.c.
.3 sshPass.h.
.3 trafgen.
.3 trafgen\_configs.h.
.3 trafgenlib.h.
.3 trafgen\_wrapper.c.
.3 trafgen\_wrapper.h.
.2 dosgen-ansible.
.3 ansible.cfg.
.3 hosts.
.3 ntp.yml.
.3 roles.
.4 ntp.
.4 snmp.
.3 snmp.yml.
.2 dosgen.spec.
.2 dosgen-tools.
.2 dosgen-web.
.3 bin.
.4 www.
.3 cert.pem.
.3 config.json.
.3 dosgen-web.env.
.3 dosgen-web.init.
.3 dosgen-web.service.
.3 key.pem.
.3 Makefile.
.3 package.json.
.3 passport.js.
.3 public.
.4 fonts.
.4 images.
.4 javascripts.
.4 stylesheets.
.3 routes.
.4 auth.js.
.4 dosgen.js.
.3 server.js.
.3 views.
.4 dosgen.ejs.
.4 error.ejs.
.4 login.ejs.
.2 machine-provisioning.
.3 ntpserver.
.4 fedora14.
.5 fedora14.ks.
.5 installation\_command.
.2 Makefile.
.2 man.
.3 dosgen.8.ronn.
.3 Makefile.
.3 README.md.
.2 README.md.
}
}