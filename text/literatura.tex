% Pro sazbu seznamu literatury použijte jednu z následujících možností

%%%%%%%%%%%%%%%%%%%%%%%%%%%%%%%%%%%%%%%%%%%%%%%%%%%%%%%%%%%%%%%%%%%%%%%%%
%1) Seznam citací definovaný přímo pomocí prostředí literatura / thebibliography

\begin{literatura}{99}

\bibitem{pirati_internet}
Parlament 2017 - Dlouhodobý program - Internet. \textit{Pirátská strana} [online]. [cit. 2017-11-19]. Dostupné z: https://www.pirati.cz/program/dlouhodoby/internet/

\bibitem{Jirasek2012}
JIRÁSEK, Petr, Luděk NOVÁK a Josef POŽÁR. \textit{Výkladový slovník kybernetické bezpečnosti}. Praha: Policejní akademie ČR v Praze, 2012. ISBN 978-80-7251-378-9.

\bibitem{akamai_q2_2017}
State of the Internet / security Q2 2017 Report. In: ARTEAGA, Jose, Dave LEWIS, Chad SEAMAN, et al. \textit{State of the Internet: Quarterly Security Reports} [online]. 2017, s.~2 [cit. 2017-11-21]. Dostupné z: https://www.akamai.com/uk/en/multimedia/documents/state-of-the-internet/q2-2017-state-of-the-internet-security-report.pdf


\end{literatura}


%%%%%%%%%%%%%%%%%%%%%%%%%%%%%%%%%%%%%%%%%%%%%%%%%%%%%%%%%%%%%%%%%%%%%%%%%
%%2) Seznam citací pomocí BibTeXu
%% Při použití je nutné v TeXnicCenter ve výstupním profilu aktivovat spouštění BibTeXu po překladu.
%% Definice stylu seznamu
%\bibliographystyle{unsrturl}
%% Pro českou sazbu lze použít styl czechiso.bst ze stránek
%% http://www.fit.vutbr.cz/~martinek/latex/czechiso.tar.gz
%%\bibliographystyle{czechiso}
%% Vložení souboru se seznamem citací
%\bibliography{text/literatura}
%
%% Následující příkaz je pouze pro ukázku sazby literatury při použití BibTeXu.
%% Způsobí citaci všech zdrojů v souboru odkazy.bib, i když nejsou citovány v textu.
%\nocite{*}