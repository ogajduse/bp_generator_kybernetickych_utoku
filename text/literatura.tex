% Pro sazbu seznamu literatury použijte jednu z následujících možností

%%%%%%%%%%%%%%%%%%%%%%%%%%%%%%%%%%%%%%%%%%%%%%%%%%%%%%%%%%%%%%%%%%%%%%%%%
%1) Seznam citací definovaný přímo pomocí prostředí literatura / thebibliography

\begin{literatura}{99}

\bibitem{pirati_internet}
Parlament 2017 - Dlouhodobý program - Internet. \textit{Pirátská strana} [online]. [cit. 2017-11-19]. Dostupné z: https://www.pirati.cz/program/dlouhodoby/internet/

\bibitem{Jirasek2012}
JIRÁSEK, Petr, Luděk NOVÁK a Josef POŽÁR. \textit{Výkladový slovník kybernetické bezpečnosti}. Praha: Policejní akademie ČR v~Praze, 2012. ISBN 978-80-7251-378-9.

\bibitem{akamai_q2_2017}
State of the Internet / security Q2 2017 Report. In: ARTEAGA, Jose, Dave LEWIS, Chad SEAMAN, et al. \textit{State of the Internet: Quarterly Security Reports} [online]. 2017, s.~2 [cit. 2017-11-21]. Dostupné z: https://www.akamai.com/uk/en/multimedia/documents/state-of-the-internet/q2-2017-state-of-the-internet-security-report.pdf

\bibitem{DDoSaaS}
BUKAČ, Vít, Zdeněk ŘÍHA, Lukáš NĚMEC, Vlasta ŠŤAVOVÁ a Václav MATYÁŠ. DDoSaaS: DDoS jako služba. In IS2: From trends to solutions. Praha: Tate International, 2015. s. 35-39, 5 s. ISBN 978-80-86813-28-8.
%do BibTeXu zkopirovat rucne ze stranek univerzity	

\bibitem{TA14-017A}
Alert TA14-017A: UDP-Based Amplification Attacks. In: \textit{US-CERT} [online]. 2014 [cit. 2017-11-26]. Dostupné z: https://www.us-cert.gov/ncas/alerts/TA14-017A

\bibitem{CVE-2013-5211}
CVE-2013-5211. \textit{CVE - Common Vulnerabilities and Exposures} [online]. 15.8.2013 [cit. 2018-05-27]. Dostupné z: http://cve.mitre.org/cgi-bin/cvename.cgi?name=CVE-2013-5211

\bibitem{RHBZ1047854}
GUPTA, Ratul. CVE-2013-5211 ntp: DoS in monlist feature in ntpd. \textit{Red Hat Bugzilla} [online]. 2.1.2014 [cit. 2018-05-27]. Dostupné z: https://bugzilla.redhat.com/show\_bug.cgi?id=1047854

\bibitem{rfc1157}
CASE, Jeffrey D., DAVIN J, Martin Lee SCHOFFSTALL a James R. DAVIN. \textit{Simple Network Management Protocol (SNMP): RFC 1157} [online]. May 1990, , 36 [cit. 2018-05-22]. DOI: 10.17487/RFC1157. Dostupné z: https://tools.ietf.org/html/rfc1157

\bibitem{ntpbug1532}
Bug 1532 - remove ntpd support for ntpdc's monlist (use ntpq's mrulist). \textit{NTP Bugzilla} [online]. 2010 [cit. 2018-05-26]. Dostupné z: http://bugs.ntp.org/show\_bug.cgi?id=1532

\bibitem{Halaska2016}
HALAŠKA, Peter. \textit{Generátor kybernetických útoků} [online]. Brno: Vysoké učení technické v Brně. Fakulta elektrotechniky a komunikačních technologií, 2016 [cit. 2018-05-27]. Dostupné z: http://hdl.handle.net/11012/59938. Diplomová práce. Vysoké učení technické v Brně. Fakulta elektrotechniky a komunikačních technologií. Ústav telekomunikací. Vedoucí práce Jan Hajný.

\bibitem{Gregr2017}
GREGR, Filip. \textit{Generátor kybernetických útoků} [online]. Brno: Vysoké učení technické v Brně. Fakulta elektrotechniky a komunikačních technologií, 2017 [cit. 2018-05-27]. Dostupné z: http://hdl.handle.net/11012/65884. Diplomová práce. Vysoké učení technické v Brně. Fakulta elektrotechniky a komunikačních technologií. Ústav telekomunikací. Vedoucí práce Jan Hajný.

\bibitem{Stange-snmp-amplification}
STANGE, Kevin. \textit{Preventing SNMP Amplication Attacks} [online]. 23 March 2015 [cit. 2018-05-27]. Dostupné z: https://support.steadfast.net/knowledgebase/article/View/110/0/preventing-snmp-amplication-attack

\bibitem{Kukla2010}
KUKLA, Miloš. \textit{Systém pro získávání provozních údajů o počítačové síti} [online]. Brno: Vysoké učení technické v Brně. Fakulta informačních technologií, 2010 [cit. 2018-05-27]. Dostupné z: http://hdl.handle.net/11012/54397. Diplomová práce. Vysoké učení technické v Brně. Fakulta informačních technologií. Ústav informačních systémů. Vedoucí práce Petr Matoušek.

\bibitem{Skrivanek2006}
SKŘIVÁNEK, Lukáš. \textit{36MPS - SNMP: Simple Network Management Protocol} [online]. K336 FEL ČVUT, Karlovo nám. 13, 121 35 Praha 2, 2006 [cit. 2018-05-27]. Dostupné z: http://skriv.wz.cz/MPS/SNMP/index.html

\bibitem{Odom2009}
ODOM, Wendell, Rus HEALY a Naren MEHTA. \textit{Směrování a přepínání sítí: autorizovaný výukový průvodce}. Brno: Computer Press, 2009. Samostudium. ISBN 978-80-251-2520-5.

\bibitem{Kopecky2007}
KOPECKÝ, Martin. \textit{Technologie pro snadnou konfiguraci sítě klienta a service discovery} [online]. Brno, 2007 [cit. 2018-05-28]. Dostupné z: https://theses.cz/id/d8odmo. Bakalářská práce. Masarykova univerzita, Fakulta informatiky. Vedoucí práce Vlastimil Holer.

\bibitem{Deering1989}
DEERING, Steve. \textit{RFC 1112: Host Extensions for IP Multicasting} [online]. 1989 [cit. 2018-05-29]. Dostupné z: https://tools.ietf.org/html/rfc1112

\end{literatura}


%%%%%%%%%%%%%%%%%%%%%%%%%%%%%%%%%%%%%%%%%%%%%%%%%%%%%%%%%%%%%%%%%%%%%%%%%
%%2) Seznam citací pomocí BibTeXu
%% Při použití je nutné v TeXnicCenter ve výstupním profilu aktivovat spouštění BibTeXu po překladu.
%% Definice stylu seznamu
%\bibliographystyle{unsrturl}
%% Pro českou sazbu lze použít styl czechiso.bst ze stránek
%% http://www.fit.vutbr.cz/~martinek/latex/czechiso.tar.gz
%%\bibliographystyle{czechiso}
%% Vložení souboru se seznamem citací
%\bibliography{text/literatura}
%
%% Následující příkaz je pouze pro ukázku sazby literatury při použití BibTeXu.
%% Způsobí citaci všech zdrojů v souboru odkazy.bib, i když nejsou citovány v textu.
%\nocite{*}