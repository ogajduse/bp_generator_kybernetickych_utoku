% !TeX spellcheck = cs_CZ
\chapter{Závěr}
V této práci se zabývám rozšířením generátoru kybernetických útoků o nové útoky. V úvodu této práce jsou definovány kybernetické útoky obecně a útoky, které následně implementuji, jsou popsány detailněji. Práce detailně popisuje typy kybernetických útoků, jejich různé formy a dělení. Je vysvětlen cíl kybernetických útoků, kdo za nimi stojí a jakým způsobem jsou vykonávány.

V druhé části této práce se zabývám samotným popisem rozšíření nástroje DoSgen, popisuji jeho strukturu a možnosti použití pro mnou zvolené útoky. Každý typ útoku je popsán ve vlastní kapitole, přičemž je zmíněna konkrétní vlastnost nebo zranitelnost síťového protokolu využitého při útoku případně jsou zmíněny chyby v nastavení infrastruktury, které umožňují sestavení daného útoku. Vždy je zmíněn způsob, jakým lze tomuto útoku zabránit.

V další části práce se zabývám popisem implementace konkrétních útoků a jejich použitím. Jelikož implementované útoky vyžadují určitou konfiguraci stanic, které participují na útoku, jsou zde také zmíněny prerekvizity k jednotlivým implementovaným útokům. Všechny popisované útoky lze kategorizovat jako útoky zesílené, přesněji útoky reflektované s amplifikací, přičemž vždy na útoku participují tři nebo více entit. Toto odlišuje mnou implementované útoky od již implementovaných útoků předchozími dvěma pracemi.

V poslední části této kapitoly zmiňuji rozšíření webového rozhraní nástroje DoSgen o nové útoky. Práce na této části nástroje obnášela i úpravu komponent zajišťující spouštění webové aplikace. Byly také aktualizovány komponenty, na nichž je webové rozhraní závislé, z důvodu nálezu zranitelností v nich.

V poslední kapitole se věnuji testování nově implementovaných útoků. V první části kapitoly je představena testovací síť a její parametry. Dále jsou prezentovány výsledky testování propustnosti celé sítě.

Každý útok způsobil zvýšený provoz generovaný od amplifikátoru k oběti. Lze tedy tvrdit, že útoky byly do aplikace DoSgen implementovány správně. Tyto implementované útoky tak lze použit k vyčerpání síťových prostředků oběti. Čím více amplifikátorů je útočník schopen využít, tím účinnější je tento útok.

Při testování jsem narazil na problém s virtuálním switchem hypervizoru, který při generování útoku nástrojem DoSgen, způsobuje vyšší odezvu i mezi stanicemi, které neparticipují na útoku. Z tohoto lze usoudit, že pro testování tohoto útoku je lepší fyzický switch, který dokáže zátěž rozložit tak, aby stanice, které do útoku nejsou zapojeny, nebyly tímto ovlivněny.

Popis testování jednotlivých útoků vhodně doplňují obrázky a grafy.
Ani v jednom případě nedošlo k DoS serveru oběti. Při provádění útoků se však v každém z případů zvýšila doba odezvy přibližně 8krát. Popis testování jednotlivých útoků vhodně doplňují grafy a obrázky z průběhu testování.
