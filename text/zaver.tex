% !TeX spellcheck = cs_CZ
\chapter{Závěr}
V této práci se zabývám rozšířením generátoru kybernetických útoků o nové útoky. V úvodu této práce jsou definovány kybernetické útoky obecně, útoky, které následně implementuji, jsou popsány detailněji. Práce detailně popisuje typy kybernetických útoků, jejich různé formy a dělení. Je vysvětlen cíl kybernetických útoků, kdo za nimi stojí a jakým způsobem jsou vykonávany.

V druhé části této se zabývám samotným popisem rozšíření nástroje DoSgen, popisuji jeho strukturu a možnosti použití pro mnou zvolené útoky. Dále se zabývám popisem implementace konkrétních útoků a jejich použitím.

V poslední kapitole se věnuji testování nově implementovaných útoků. Jsou přiloženy snímky obrazovky z průběhu testování.

Ve výsledku se mi podařilo implementovat a řádně otestovat pouze jeden útok, jelikož jsem neodhadl komplexnost této práce.

Dále se plánuji zabývat implementaci zbylých nedokončených útoků, konkrétně SNMP a SMTP. Jedním z dalších mých cílů je restrukturalizace kódu, aktualizace nástroje trafgen implementovaného v DoSgenu a vytvoření manuálové stránky pro DoSgen. V neposlední řadě plánuji vytvořit RPM balíček pro Red Hat Linuxové distribuce.