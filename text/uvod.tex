% !TeX spellcheck = cs_CZ
\chapter*{Úvod}
\phantomsection
\addcontentsline{toc}{chapter}{Úvod}


V~dnešní době, kdy existují jisté iniciativy, které chtějí prosadit internet jako právo, je internet 
pro moderního člověka, jeho denní činnosti, i firmy prakticky neodmyslitelný \cite{pirati_internet}. 
Dotčeny jsou nejen oblasti podnikání ale také oblasti vzdělání, umění a další. Výpadek této služby má 
mnohdy nepředstavitelné následky jako finanční ztrátu nebo mnohem citelnější jako nefunkčnost eshopů, 
bankovních či jiných informačních systémů. Za těmito výpadky můžeme nejčastěji hledat kybernetický útok, 
což je promyšlené škodlivé jednání útočníků zaměřené na  IT  infrastrukturu  za  účelem  způsobit 
poškození  a  získat  citlivé  či  strategicky  důležité 
informace promyšlené nebo naplnit určitý sociální, ideologický, náboženský, politický nebo jiný záměr 
\cite{Jirasek2012}. Čím více je fungování společnosti na internetu závislejší, tím více jsou tyto hrozby 
závažnější a je třeba se jimi zabývat. Nejčastějším typem útoku je \zkratka{zk_dos}, proti kterému 
v~mnoha případech neexistuje  žádná spolehlivá obrana \cite{akamai_q2_2017}. Útoky tohoto typu mají 
za cíl znepřístupnit danou službu běžící na serveru nebo server samotný, pro běžného uživatele se tak 
tváří býti nedostupný. Útočníkem se může stát kdokoli, ať už je to běžný uživatel, síťový expert nebo 
v~také organizovaná skupina útočníků. Právě tyto skupiny jsou schopny docílit většího efektu a jsou 
schopny napáchat větší škody. Cíl útoku nemusí nutně být národního či nadnárodního rozměru.

Je zjevné, že kybernetickou bezpečnost nelze podcenit a bezesporu zaujímá neméně důležitou roli. Tato 
práce se zabývá popisem \zkratkatext{zk_dos} útoků a rozšířením softwarového generátoru kybernetických útoků 
o~nové typy útoků.

První část popisuje \zk{zk_dos} útoky, přiblížím jejich původ, fungování. Implementované typy útoků 
popíši detailněji.