% !TeX spellcheck = cs_CZ
\chapter{Rozšíření nástroje DoSgen}
Úkolem tohoto semestrálního projektu bylo rozšíření software generující kybernetické útoky. Tato aplikace nese název DoSgen a je napsána v jazyce C. Původním autorem této aplikace je Peter Halaška, který ji vytvořil v rámci své diplomové práce. Tento nástroj byl rozšířen Filipem Grégrem, také coby diplomová práce. Tento autor své útoky z důvodu neproveditelnosti v rámci původní struktury implementoval v jazyce C++.

Pro implementaci v rámci své práce jsem si zvolil následující útoky:
\begin{itemize}
	\item{NTP Flood}
	\item{SMTP Flood}
	\item{SNMP Flood}
\end{itemize}

Teoretický popis těchto útoků je uveden v kapitole \ref{sec:implementovane_utoky}. V první fázi bylo na místě zvážení, zda-li je možno využít knihovny \texttt{LibDoS} při implementaci vlastních útoků. Povaha této knihovny nebrání implementaci mnou vybraných útoků.

V průběhu vývoje jsem využíval nástroje Wireshark pro sledování síťového provozu mezi útočníkem a obětí. Všechny útoky byly testovány na běžné uživatelské stanici s operačním systémem Linux, konkrétně Fedora 27. Tato stanice sloužila pouze jako generátor útoků. Stanice vystupující při útoku jako \uv{reflektor} nebo \uv{amplifikátor} a \uv{oběť} jsou plně virtualizované operační systémy skrze hypervizor kernelu (KVM) přistupující ke stejnému síťovému rozhraní \texttt{vnet0}, na něž je připojena i hostitelská stanice. Toto rozhraní je od internetu odděleno prostřednictvím NAT. Pro snazší správu virtualizovaných stanic bylo použito aplikace Virtual Machine Manager ve verzi 1.4.3.


\section{Struktura nástroje}
%dosgen, trafgen wrapper, libdos, tcpgen, tcpgen?, gui
Nástroj DoSgen se sestává z více komponent, které ve výsledku tvoří jeden celek. V následujících kapitolách jsou právě tyto popsány.

\subsection{Trafgen}
Jádrem aplikace DoSgen je program Trafgen, nástroj distribuovaný v rámci sady nástrojů netsniff-ng vyvinutou Danielem Borkmannem. Netsniff-ng je určena pro operační systémy Linux, distribuována je pod licencí GNU General Public License v2.0.

Nástroj trafgen je vysoce výkonný generátor síťového provozu napsaný v jazyce C. Využití najde při ladění sítí, testování jejich výkonnosti. Lze jej použít také pro \textit{fuzz-testing}, tedy techniku, kdy jsou na vstup testovaného programu dodávána náhodná či neočekávaná data.

Vysoké výkonnosti je u trafgenu dosaženo pomocí \textit{zero-copy} mechanizmů, při jehož použití nemusí jádro operačního systému kopírovat každý paket z prostory jádra do uživatelského prostoru a naopak. Velkou výhodou tohoto nástroje je také to, že je možno pro generování paketů využít více jader procesoru.

Trafgen umožňuje sestavení paketu prostřednictvím svého nízkoúrovňového jazyka. Proto není limitován žádným protokolem a je možné jej tedy použít jakkoli. K tomuto nástroji lze přistupovat skrze příkazový řádek, kde je možno specifikovat potřebné parametry a cestu ke konfiguračnímu souboru datového paketu.




%\section{Konfigurační jazyk nástroje Trafgen}
%přemístit k popisu trafgenu?

\section{Implementace útoků}
%konkrétně každý útok


\section{Instalace nástroje}
%docker


\section{Použití nástroje}
