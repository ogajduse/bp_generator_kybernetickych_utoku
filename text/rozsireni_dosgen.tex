% !TeX spellcheck = cs_CZ
\chapter{Rozšíření nástroje DoSgen}
Úkolem tohoto semestrálního projektu bylo rozšíření software generující kybernetické útoky. Tato aplikace nese název DoSgen a je napsána v jazyce C. Původním autorem této aplikace je Peter Halaška, který ji vytvořil v rámci své diplomové práce. Tento nástroj byl rozšířen Filipem Gregrem, také coby diplomová práce. Tento autor své útoky z důvodu neproveditelnosti v rámci původní struktury implementoval v jazyce C++.

Pro implementaci v rámci své práce jsem si zvolil následující útoky:
\begin{itemize}
	\item{NTP Flood}
	\item{SMTP Flood}
	\item{SNMP Flood}
\end{itemize}

Teoretický popis těchto útoků je uveden v kapitole \ref{sec:implementovane_utoky}. V první fázi bylo na místě zvážení, zda-li je možno využít knihovny \texttt{LibDoS} při implementaci vlastních útoků. Výsledným zjištěním bylo, že povaha této knihovny nebrání implementaci mnou vybraných útoků.

Při vývoji bylo použito prostředí zmíněného v kapitole \ref{chap:testovani}.

\section{Struktura nástroje}
%dosgen, trafgen wrapper, libdos, tcpgen, tcpgen?, gui
Nástroj DoSgen se sestává z více komponent, které ve výsledku tvoří jeden celek. V následujících kapitolách jsou právě tyto popsány.

Samotný dosgen se stará o převzetí vstupních argumentů od uživatele. Implementuje dodatečné funkce pro jejich kontrolu a zabezpečuje formátování pro další použití a jiná opatření.

\subsection{Trafgen}
Jádrem aplikace DoSgen je program Trafgen, nástroj distribuovaný v rámci sady nástrojů netsniff-ng vyvinutou Danielem Borkmannem. Netsniff-ng je určena pro operační systémy Linux, distribuována je pod licencí GNU General Public License v2.0.

Nástroj trafgen je vysoce výkonný generátor síťového provozu napsaný v jazyce C. Využití najde při ladění sítí, testování jejich výkonnosti. Lze jej použít také pro \textit{fuzz-testing}, tedy techniku, kdy jsou na vstup testovaného programu dodávána náhodná či neočekávaná data.

Vysoké výkonnosti je u trafgenu dosaženo pomocí \textit{zero-copy} mechanizmů, při jehož použití nemusí jádro operačního systému kopírovat každý paket z prostoru jádra do uživatelského prostoru a naopak. Velkou výhodou tohoto nástroje je také to, že je možno pro generování paketů využít více jader procesoru.

Trafgen umožňuje sestavení paketu prostřednictvím svého nízkoúrovňového jazyka. Proto není limitován žádným protokolem a je možné jej tedy použít jakkoli. K tomuto nástroji lze přistupovat skrze příkazový řádek, kde je možno specifikovat potřebné parametry a cestu ke konfiguračnímu souboru datového paketu.

V této práci je implementovaný trafgen z netsniff-ng ve verzi 0.6.1. Aktuální verze je 0.6.3.

\subsection{LibDoS}
Knihovna LibDoS, kterou DoSgen využívá, sdružuje knihovnu vytvořenou z nástroje trafgen spolu s rozhraním \texttt{Trafgen wrapper}.

\subsection{Trafgen wrapper}
Hlavním úkolem modulu \texttt{Trafgen wrapper} je shromáždění potřebných parametrů pro jejich předání nástroji trafgen, ve výsledku tedy úspěšné spuštění útoku. Trafgen na vstupu očekává soubor s konfigurací paketu. Ten je taktéž vytvořen tímto modulem, který jej sestaví ze souboru \texttt{trafgen\_configs.h}, v němž se nachází pro každý útok šablony, do kterých se doplní informace získané od uživatele.

%TODO \section{Konfigurační jazyk nástroje Trafgen}
%přemístit k popisu trafgenu?

\section{Instalace nástroje}
%docker
V této kapitole bude popsána instalace nástroje DoSgen na operačním systému Fedora. Jsou zde zmíněny knihovny, jež je třeba instalovat pro jeho plnou funkčnost. 

\noindent Instalace následujících knihoven je nutností, jelikož jsou přímými závislostmi nástroje trafgen.
\begin{lstlisting}
dnf -y install flex bison libnl3-devel libssh-devel
\end{lstlisting}

\noindent Dále je třeba nainstalovat nástroje potřebné ke kompilaci kódu nástroje.
\begin{lstlisting}
dnf -y install make pkg-config gcc-c++
\end{lstlisting}

\noindent Instalace následujících balíčků je pouze volitelné, jde o užitečné nástroje, které najdou své místo vedle nástroje DoSgen.
\begin{lstlisting}
dnf -y install tcptrack nmap httping john iftop nload \
iputils net-tools
\end{lstlisting}

\noindent Dále je potřeba přeložit zdrojový kód. Provedeme příkazem make:
\begin{lstlisting}
cd dosgen/dosgen
make
\end{lstlisting}

V rámci této práce byl vytvořen také Dockerfile, pomocí kterého lze sestavit kontejner s nainstalovaným nástrojem DoSgen. Výhoda tohoto kontejneru spočívá v tom, že procesy spuštěné v kontejneru jsou oddělené od procesů v hostitelské stanici. Tento kontejner staví na operačním systému Fedora ve verzi 27. Sestavení kontejneru a jeho spuštění lze docílit následovně.
\begin{lstlisting}
docker build -t <NÁZEV_KONTEJNERU> .
docker run <NÁZEV_KONTEJNERU>
\end{lstlisting}

\section{Implementace útoků}
Původní struktura nástroje zůstala stejná, stávající moduly byly upraveny pro funkčnost mnou implementovaných útoků.

\subsection{NTP Flood}

Prerekvizitou tohoto útoku je 600 klientů, zapsaných v databázi NTP serveru. Toto nelze nasimulovat z jedné uživatelské stanice běžným způsobem. Proto jsem přistoupil k tvorbě nástroje v jazyce Python, pomocí kterého si nasimuluji množství klientů požadujících po NTP serveru aktualizaci času. Tento skript využívá knihoven \texttt{scapy} pro sestavení paketu a \texttt{faker} pro generování náhodné IP adresy.

Dalším krokem bylo umístění konfigurace paketu do hlavičkového souboru \newline \texttt{trafgen\_configs.h}. Do této konfigurace si následně DoSgen doplní zdrojovou a cílovou adresu IP. 
V dalším kroku byl do souboru \texttt{dosgen.c} doplněn nový přepínač pro tento útok. Při jeho použití je zavolána funkce \texttt{ntp\_flood} z modulu \texttt{argsParse.c}, která zajišťuje validaci vstupních parametrů. Při kladném výsledku validace je zavolána funkce \texttt{prepare\_ntp}, která upravuje konfigurační soubor. Následně je zavolán samotný trafgen a útok je spuštěn.

Teoretický popis tohoto útoku je popsán v kapitole \ref{subsec:ntp_flood}.


\section{Použití nástroje}
Pro spuštění NTP útoku pomocí nástroje dosgen využijme následující příkaz.
\begin{lstlisting}
	./dosgen -i virbr0 -P 4 --ntp -s 192.168.124.129 \
                     -d 192.168.124.14
\end{lstlisting}

Za parametrem \texttt{-i} následuje název rozhraní, skrze ktere je provoz následně generován. Parametr \texttt{-P} určuje počet jader, kolik je využito pro generování provozu, za ním následuje \texttt{--ntp} znamenající, že z dostupných útoků bude vybrán útok NTP Flood. Parametr \texttt{-s} označuje zdrojovou IP adresu, která bude uvedena v paketu, ve skutečnosti jde o IP adresu oběti. Poslední \texttt{-d} značí IP adresu, na kterou je dotaz zaslán.

