% !TeX spellcheck = cs_CZ
\chapter[DoS útoky]{DoS útoky}
V této kapitole budou rozebrány \zk{zk_dos} útoky, jejich princip, charakteristika a následně dělení. Dále tato kapitola zmiňuje využití \zk{zk_ddos} útoků jako služby. V závěru jsou zmíněny implementované útoky.

\section{Charakteristika}
\zk{zk_dos} lze charakterizovat jako kybernetický útok, který si klade za cíl znepřístupnění cílové služby legitimním uživatelům. Využívají jedné nebo více zranitelností v implementaci konkrétního software. Ve své podstatě každá stanice se může stát cílem, obětí, útoku \zk{zk_dos}. \zk{zk_dos} útoky se mezi útočníky velmi rychle staly populárními a tento trend se nemění \cite{akamai_q2_2017}. Realizace \zk{zk_dos} útoku není nikterak obtížná a nevyžaduje z útočníkovy strany hluboké znalosti, jelikož nástroje k jejich provedení jsou běžně dostupné, viz. kapitola \ref{chap:nastroje_pro_dos}. Fakt, že narušit činnost sítě bývá jednodušší, než do ní získat přístup podporuje tento trend a jejich rozšířenost. 

Motivy takových útoků mohou být hacktivizmus, vydírání, nekalé konkurenční praktiky, zlomyslnost nebo smokescreen coby prostředek krytí jiného současně prováděného útoku nebo určitá forma protestu. Dalšími motivy mohou být také uznání hackerské komunity nebo také politické politický motiv.

Velká část útoků využívá nedostatků v architektuře rodiny protokolů TCP/IP, které byly k tomu, aby byly funkční, nikoliv bezpečné. Používaly se v otevřeném a důvěryhodném prostředí, čemuž v dnešní době není.

Trend naznačuje rostoucí počet druhu a implementací jednotlivých útoků, proto bychom měli věnovat pozornost a určité úsilí na ochranu proti nim. Útočník je ve výhodě, jelikož nelze jednoznačně predikovat cíl jeho útoku.


\section{Typy útoků}
\subsection{Útok z jednoho zdroje a útok distribuovaný}
\zk{zk_dos} útoky mohou být prováděny z jednoho místa, což je méně časté, jelikož při \zk{zk_dos} útoku není útočník z důvodu malé šířky pásma schopen vygenerovat dostatečný síťový provoz na to, aby způsobil odepření služby na cíli svého útoku. Lze tedy jeho účinek znásobit a to použitím většího počtu útočících stanic. Takový útok nazýváme \zkratka{zk_ddos}. V případě \zk{zk_dos} útoku není útočník mnohdy schopen vygenerovat dostatečný síťový provoz na to, aby způsobil odepření služby na cíli svého útoku.

Pro distribuovaný \zk{zk_dos} útok je charakteristický velmi vysoký počet uživatelských stanic, či serverů, použitých k realizaci tohoto útoku. Pohybujeme se v řádech desítek, stovek až milionů počítačů. 
Často jsou tyto počítače napadeny virem, trojským koněm nebo programem, který v sobě ukrývá skrytou funkci a následně jsou zneužitý k účelům \zk{zk_dos} útoku bez vědomí jejich majitele. Zajímavým cílem útočníků jsou zařízení IoT, které útočník může následně využít k \zk{zk_dos} útoku.

Takové stanice nazýváme \uv{boti} a seskupení více těchto stanic \uv{botnety}. Důvody pro vytvoření botnetu jsou větší šířka pásma, šíření spamu nebo redistribuce malware nejen do sítě, kde se \uv{bot} nachází. Tito \uv{boti} jsou zapojeni do \zkratka{zk_cnc} infrastruktury spolu s \zkratka{zk_irc} servery skrze které může útočník rozesílat instrukce botům. %TODO:zdroj
\zk{zk_irc} servery poskytují centralizovanou správu botů nebo celých botnetů skrze \zk{zk_cnc} mechanizmy.
Útoky vedené skrze tyto botnety dokáží plně zahltit šířku pásma oběti a způsobí tak odepření služby. 
Botneti mohou být rozmístěni kdekoli po světě, což znesnadňuje ochranu tím, že nelze jednoznačně podle geolokační stopy IP adresy filtrovat provoz z konkrétní země, což je podpořeno možností měnit zdrojovou adresu v IP datagramu, tzv. IP spoofing.
%TODO:obrazek utoku za pomoci spoofovane ip adresy
%TODO:zde vlozit obrazek ddos utoku

\subsection{Zesílené útoky}
K zesílení útoku lze použít \uv{reflektor}. Reflektor je prvek v síti, který reflektuje provoz iniciovaný ze strany útočníka. Při takovém útoku je \uv{spoofována} zaměněna zdrojovoá adresa v IP datagramu. Útočník při trojcestném handshakingu (three-way handshake) zasílá na reflektor paket s příznakem SYN se spoofovanou adresou, ten následně odpoví paketem s příznakem SYN/ACK na podvrženou IP adresu, tedy adresou oběti. Reflektor tak neodesílá odpověď útočníkovi ale oběti. Při dostatečném počtu odeslaných žádostí směrem k oběti dochází k vyčerpání zdrojů, tedy k odepření služby. Oběť se tak může domnívat, že na ni útočí reflektory. Pravý útočník zůstává skryt. Reflektovaný útok lze označit jako \zkratka{zk_drdos}.

Jiným typem jsou takzvané \uv{amplifikované} útoky za použití \uv{amplifikátoru}. \uv{Amplifikátor} amplifikuje, tedy zvětšuje síťový provoz zaslaný útočníkem a zasílá jej na spoofovanou IP adresu oběti. Útočník je díky tomuto schopen vygenerovat dostatečný síťový provoz pro zahlcení oběti.

%TODO:tabulka z https://www.us-cert.gov/ncas/alerts/TA14-017A nebo https://en.wikipedia.org/wiki/Denial-of-service_attack#Reflected_.2F_spoofed_attack

Pro větší účinek těchto zesílených útoků lze využít namísto \zk{zk_dos} útoku útok \zk{zk_ddos}.

\section{Dělení útoků}
%Útoky lze také specifikovat a rozdělit podle TCP/IP vrstev, skrze které působí.
\subsection{Podle počtu útočících zařízení}

\subsection{Podle spotřeby zdrojů}

\subsection{Podle rychlosti}

\subsection{Podle spoje}
%spojované a nespojované

\section{DDoS jako služba}

\section{Útoky mířené na síťové zdroje}

\subsection{Záplavové}

\section{Útoky mířené na serverové zdroje}
% jak funguje normalne, slabina v tcp/ip, pomale utoky

\section{Útoky mířené na aplikační zdroje}
%http
%%vice druhu http utoku
%slow loris?
%slow read?

\section{Implementované útoky}

