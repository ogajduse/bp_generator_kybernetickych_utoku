% !TeX spellcheck = cs_CZ
\chapter[DoS útoky]{DoS útoky}
V~této kapitole budou rozebrány \zk{zk_dos} útoky, jejich princip, charakteristika a následně
dělení. Dále tato kapitola zmiňuje využití \zk{zk_ddos} útoků jako služby. V~závěru jsou
zmíněny implementované útoky.

\section{Charakteristika}
\zk{zk_dos} lze charakterizovat jako kybernetický útok, který si klade za cíl znepřístupnění
 cílové služby legitimním uživatelům. Využívají jedné nebo více zranitelností v~implementaci
 konkrétního software. Ve své podstatě každá stanice se může stát cílem, obětí, útoku
 \zk{zk_dos}. \zk{zk_dos} útoky se mezi útočníky velmi rychle staly populárními a tento trend
 se nemění \cite{akamai_q2_2017}. Realizace \zk{zk_dos} útoku není nikterak obtížná a
 nevyžaduje z~útočníkovy strany hluboké znalosti, jelikož nástroje k~jejich provedení jsou
 běžně dostupné, viz. kapitola \ref{chap:nastroje_pro_dos}. Fakt, že narušit činnost sítě bývá
 jednodušší, než do ní získat přístup podporuje tento trend a jejich rozšířenost. 

Motivy takových útoků mohou být hacktivizmus, vydírání, nekalé konkurenční praktiky,
zlomyslnost nebo smokescreen coby prostředek krytí jiného současně prováděného útoku nebo
určitá forma protestu. Dalšími motivy mohou být také uznání hackerské komunity nebo také
politické politický motiv.

Velká část útoků využívá nedostatků v~architektuře rodiny protokolů TCP/IP, které byly k~tomu,
aby byly funkční, nikoliv bezpečné. Používaly se v~otevřeném a důvěryhodném prostředí, čemuž
v~dnešní době není.

Trend naznačuje rostoucí počet druhu a implementací jednotlivých útoků, proto bychom měli
věnovat pozornost a určité úsilí na ochranu proti nim. Útočník je ve výhodě, jelikož nelze
jednoznačně predikovat cíl jeho útoku.


\section{Typy útoků}
\subsection{Útok z~jednoho zdroje a útok distribuovaný}
\zk{zk_dos} útoky mohou být prováděny z~jednoho místa, což je méně časté, jelikož při
\zk{zk_dos} útoku není útočník z~důvodu malé šířky pásma schopen vygenerovat dostatečný síťový
provoz na to, aby způsobil odepření služby na cíli svého útoku. Lze tedy jeho účinek znásobit a
to použitím většího počtu útočících stanic. Takový útok nazýváme \zkratka{zk_ddos}. V~případě
\zk{zk_dos} útoku není útočník mnohdy schopen vygenerovat dostatečný síťový provoz na to, aby
způsobil odepření služby na cíli svého útoku.

Pro distribuovaný \zk{zk_dos} útok je charakteristický velmi vysoký počet uživatelských stanic,
či serverů, použitých k~realizaci tohoto útoku. Pohybujeme se v~řádech desítek, stovek až
milionů počítačů. 
Často jsou tyto počítače napadeny virem, trojským koněm nebo programem, který v~sobě ukrývá
skrytou funkci a následně jsou zneužitý k~účelům \zk{zk_dos} útoku bez vědomí jejich majitele.
Zajímavým cílem útočníků jsou zařízení IoT, které útočník může následně využít k~\zk{zk_dos}
útoku.

Takové stanice nazýváme \uv{boti} a seskupení více těchto stanic \uv{botnety}. Důvody pro
vytvoření botnetu jsou větší šířka pásma, šíření spamu nebo redistribuce malware nejen do sítě,
kde se \uv{bot} nachází. Tito \uv{boti} jsou zapojeni do \zkratka{zk_cnc} infrastruktury spolu
s~\zkratka{zk_irc} servery skrze které může útočník rozesílat instrukce botům. %TODO:zdroj
\zk{zk_irc} servery poskytují centralizovanou správu botů nebo celých botnetů skrze \zk{zk_cnc}
mechanizmy. Jedním z mnoha trojských koní je Linux.Kaiten. Jedná se o klienta ovládaného skrze
\zk{zk_irc} server, který může stahovat a spouštět soubory, provádět \zk{zk_ddos} útoky za
použití UDP Flood a SYN Flood útoků a také umožňuje spoofing IP adresy.

Útoky vedené skrze tyto botnety dokáží plně zahltit šířku pásma oběti a způsobí tak odepření
služby. 
Boti mohou být rozmístěni kdekoli po světě, což znesnadňuje ochranu tím, že nelze
jednoznačně podle geolokační stopy IP adresy filtrovat provoz z~konkrétní země, což je
podpořeno možností měnit zdrojovou adresu v~IP datagramu, tzv. IP spoofing.
%TODO:obrazek utoku za pomoci spoofovane ip adresy
%TODO:zde vlozit obrazek ddos utoku

\subsection{Zesílené útoky}
\subsubsection{Reflektované útoky}
\label{subsec:reflektovane_utoky}
K~zesílení útoku lze použít \uv{reflektor}. Reflektor je prvek v~síti, který reflektuje provoz
iniciovaný ze strany útočníka. Při takovém útoku je \uv{spoofována} zaměněna zdrojovoá adresa
v~IP datagramu. Útočník při trojcestném handshakingu (three-way handshake) zasílá na reflektor
paket s~příznakem SYN se spoofovanou adresou, ten následně odpoví paketem s~příznakem SYN/ACK
na podvrženou IP adresu, tedy adresou oběti. Reflektor tak neodesílá odpověď útočníkovi ale
oběti. Při dostatečném počtu odeslaných žádostí směrem k~oběti dochází k~vyčerpání zdrojů, tedy
k~odepření služby. Oběť se tak může domnívat, že na ni útočí reflektory. Pravý útočník zůstává
skryt. Reflektovaný útok lze označit jako \zkratka{zk_drdos}.

\subsubsection{Amplifikované útoky}
Jiným typem jsou takzvané \uv{amplifikované} útoky, které využívají \uv{amplifikátoru}.
\uv{Amplifikátor} amplifikuje, tedy zvětšuje síťový provoz zaslaný útočníkem a zasílá jej na
spoofovanou IP adresu oběti, což tento útok činí silnějším než je přímý útok. Útočník je díky
tomuto schopen vygenerovat dostatečný síťový provoz pro zahlcení oběti.

Příkladem takové amplifikace může být například dotaz na DNS server, v~jehož zóně se nachází
rozsáhlý TXT záznam. Nebo v~případě \zkratka{zk_ntp} serveru si můžeme zažádat o~posledních~600
adres, které se k~němu připojily. Následující tabulka \ref{tab:udp_ampl} zobrazuje,
kolikanásobně je větší odpověď od reflektoru oproti vyslané žádosti směrem k~němu.

\begin{table}[]
	\centering
	\caption{Amplifikační útoky založené na UDP \cite{TA14-017A}}
	\label{tab:udp_ampl}
	\begin{tabular}{|l|l|}
		\hline
		Protokol               & Faktor zvětšení šířky pásma    \\ \hline
		NTP                    & 556.9                          \\ \hline
		CharGen                & 358.8                          \\ \hline
		DNS                    & do 179                         \\ \hline
		QOTD                   & 140.3                          \\ \hline
		Quake Network Protocol & 63.9                           \\ \hline
		BitTorrent             & 4.0 - 54.3                     \\ \hline
		SSDP                   & 30.8                           \\ \hline
		Kad                    & 16.3                           \\ \hline
		SNMPv2                 & 6.3                            \\ \hline
		Steam Protocol         & 5.5                            \\ \hline
		NetBIOS                & 3.8                            \\ \hline
	\end{tabular}
\end{table}

Pro větší účinek těchto zesílených útoků lze využít namísto \zk{zk_dos} útoku útok \zk{zk_ddos}.

\section{Dělení útoků}
%Útoky lze také specifikovat a rozdělit podle TCP/IP vrstev, skrze které působí.
\subsection{Podle počtu útočících zařízení}

\subsection{Podle spotřeby zdrojů}

\subsection{Podle rychlosti}

\subsection{Podle spoje}
%spojované a nespojované

\section{DDoS jako služba}
Dnes uživateli běžně přístupné služby v podobě \zk{zk_dos} nebo \zk{zk_ddos} útoků poskytované
konkrétními službami lze nazvat \zkratka{zk_ddosaas}.

Definuji sebe sama jako službu stresového testování přičemž po objednateli této služby není
vyžadována znalost datových sítí. Ve smluvních podmínkách se zříkají odpovědnosti za škody
způsobené objednateli. Přičemž ani nezjišťují, zda cíl útoku, který si zákazník objednal je či
není v jeho správě. Vyžadují registraci a za úplatu poskytují stresové testování. Platbu je
možno provádět buď v Bitcoinech nebo přes PayPal. Ceny za tuto službu se pohybují přibližně v
rozmezí mezi 2-10 americkými dolary. Mnoho takových služeb nabízí i podporu typu 24/7 skrze
\zkratka{zk_im}. Dostupných bývá povětšinou pár základních útoků založených na protokolech
\zkratka{zk_tcp}, \zkratka{zk_udp} a \zkratka{zk_http}. Tyto služby využívají také IP spoofing
a zesílení útoku pomocí amplifikátorů \ref{subsec:reflektovane_utoky}. 

Největší hrozbou \zk{zk_ddosaas} útoků je jejich snadná dostupnost. Avšak tyto útoky nejsou
nikterak sofistikované a nejsou příliš velkou hrozbou pro standardně zabezpečenou síť
\cite{DDoSaaS}.

\section{Útoky mířené na síťové zdroje}

\subsection{Záplavové}

\section{Útoky mířené na serverové zdroje}
% jak funguje normalne, slabina v tcp/ip, pomale utoky

\section{Útoky mířené na aplikační zdroje}
%http
%%vice druhu http utoku
%slow loris?
%slow read?

\section{Implementované útoky}

