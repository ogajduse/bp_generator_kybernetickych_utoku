% !TeX spellcheck = cs_CZ
% Soubory musí být v kódování, které je nastaveno v příkazu \usepackage[...]{inputenc}

\documentclass[%
%  draft,    				  % Testovací překlad
  12pt,       				% Velikost základního písma je 12 bodů
  %a4paper,    				% Formát papíru je A4
	t,                  % obsah slajdů nebude centrovaný, nýbrž budou začínat od hora
	aspectratio=1610,   % poměr stran bude 16:10 (všechny projektory v učebnách na Technické 12), další volby jsou 43, 149, 169, 54, 32.
%  oneside,      			% Jednostranný tisk (výchozí)
%% Z následujicich voleb lze použít maximálně jednu:
%	dvipdfm  						% výstup bude zpracován programem 'dvipdfm' do PDF
%	dvips	  						% výstup bude zpracován programem 'dvips' do PS
%	pdftex							% překlad bude proveden programem 'pdftex' do PDF (výchozí)
	unicode,						% Záložky a informace budou v kódování unicode
% Z následujících voleb lze použít jen jednu:
%english,            % originální jazyk je angličtina
czech,              % originální jazyk je čeština (výchozí)
%slovak,             % originální jazyk je slovenčina
]{beamer}				    	% Dokument třídy 'zpráva'
%\usepackage{etex}

\usepackage[utf8]		%	Kódování zdrojových souborů je v UTF-8
	{inputenc}					% Balíček pro nastavení kódování zdrojových souborů
\usepackage{graphicx} % Balíček 'graphicx' pro vkládání obrázků
											% Nutné pro vložení log školy a fakulty

\usepackage[
	nohyperlinks				% Nebudou tvořeny hypertextové odkazy do seznamu zkratek
]{acronym}						% Balíček 'acronym' pro sazby zkratek a symbolů
											% Nutné pro použití prostředí 'seznamzkratek' balíčku 'thesis'

%% Balíček hyperref je volán třídou beamer automaticky
%\usepackage[
%	breaklinks=true,		% Hypertextové odkazy mohou obsahovat zalomení řádku
%	hypertexnames=false % Názvy hypertextových odkazů budou tvořeny
%											% nezávisle na názvech TeXu
%]{hyperref}						% Balíček 'hyperref' pro sazbu hypertextových odkazů
%											% Nutné pro použití příkazu 'nastavenipdf' balíčku 'thesis'

%\usepackage{pdfpages} % Balíček umožňující vkládat stránky z PDF souborů
                      %% Nutné při vkládání titulních listů a zadání přímo
                      %% ve formátu PDF z informačního systému

\usepackage{cmap} 		% Balíček cmap zajišťuje, že PDF vytvořené `pdflatexem' je
											% plně "prohledávatelné" a "kopírovatelné"

\usepackage{upgreek}	% Balíček pro sazbu stojatých řeckých písmem
											% např. stojaté pí: \uppi
											% např. stojaté mí: \upmu (použitelné třeba v mikrometrech)
											% pozor, grafická nekompatibilita s fonty typu Computer Modern!

%% Nastavení českého jazyka při sazbě v češtině.
\usepackage
  {babel}             % Balíček pro sazbu různojazyčných dokumentů; kompilovat (pdf)latexem!
										% převezme si z parametrů třídy správný jazyk
\usepackage{lmodern}	% vektorové fonty Latin Modern, nástupce původních Knuthových Computern Modern fontů
\usepackage{textcomp} % Dodatečné symboly
\usepackage[T1]{fontenc}  % Kódování fontu - mj. kvůli správným vzorům pro dělení slov

%\usepackage{amsmath}
\usepackage{booktabs}			% Balíček, který umožňuje v tabulce používat příkazy \toprule, \midrule, \bottomrule 


\usepackage[%
%% Z následujících voleb lze použít pouze jednu
% left,               % Rovnice a popisky plovoucich objektů budou %zarovnány vlevo
  center,             % Rovnice a popisky plovoucich objektů budou zarovnány na střed (vychozi)
%% Z následujících voleb lze použít pouze jednu
%semestral						%	sazba zprávy semestrálního projektu
bachelor						%	sazba bakalářské práce
%diploma						 % sazba diplomové práce
%treatise            % sazba pojednání o dizertační práci
%%%%%%phd                 % sazba dizertační práce
]{thesis}             % Balíček pro sazbu studentských prací
                      % Musí být vložen až jako poslední, aby
                      % ostatní balíčky nepřepisovaly jeho příkazy

\newcommand{\CC}{C\nolinebreak\hspace{-.05em}\raisebox{.4ex}{\tiny\bf +}\nolinebreak\hspace{-.10em}\raisebox{.4ex}{\tiny\bf +}}
\def\CC{{C\nolinebreak[4]\hspace{-.05em}\raisebox{.4ex}{\tiny\bf ++}}}

%%%%%%%%%%%%%%%%%%%%%%%%%%%%%%%%%%%%%%%%%%%%%%%%%%%%%%%%%%%%%%%%%
%%%%%%      Definice informací o dokumentu             %%%%%%%%%%
%%%%%%%%%%%%%%%%%%%%%%%%%%%%%%%%%%%%%%%%%%%%%%%%%%%%%%%%%%%%%%%%%


%% Název práce:
%  První parametr je název v originálním jazyce,
%  druhý je překlad v angličtině nebo češtině (pokud je originální jazyk angličtina)
\nazev{Generátor kybernetických útoků}{Cyber attack generator}

%% Jméno a příjmení autora ve tvaru
%  [tituly před jménem]{Křestní}{Příjmení}[tituly za jménem]
\autor{Ondřej}{Gajdušek}

%% Jméno a příjmení vedoucího/školitele včetně titulů
%  [tituly před jménem]{Křestní}{Příjmení}[tituly za jménem]
% Pokud osoba nemá titul za jménem, smažte celý řetězec '[...]'
\vedouci[doc.\ Ing.]{Jan}{Hajný}[Ph.D.]

%% Jméno a příjmení oponenta včetně titulů
%  [tituly před jménem]{Křestní}{Příjmení}[tituly za jménem]
% Pokud nemá titul za jménem, smažte celý řetězec '[...]'
% Uplatní se pouze v prezentaci k obhajobě;
% v případě, že nechcete, aby se na titulním snímku prezentace zobrazoval oponent, pouze jej zakomentujte;
% u obhajoby semestrální práce se oponent nezobrazuje
\oponent[doc.\ Ing.]{Jan}{Jeřábek}[Ph.D.]

%% Označení oboru studia
% První parametr je obor v originálním jazyce,
% druhý parametr je překlad v angličtině nebo češtině
\oborstudia{Informační bezpečnost}{Information security}

%% Označení fakulty
% První parametr je název fakulty v originálním jazyce,
% druhý parametr je překlad v angličtině nebo v češtině
%\fakulta{Fakulta architektury}{Faculty of Architecture}
\fakulta{Fakulta elektrotechniky a komunikačních technologií}{Faculty of Electrical Engineering and Communication}
%\fakulta{Fakulta chemická}{Faculty of Chemistry}
%\fakulta{Fakulta informačních technologií}{Faculty of Information Technology}
%\fakulta{Fakulta podnikatelská}{Faculty of Business and Management}
%\fakulta{Fakulta stavební}{Faculty of Civil Engineering}
%\fakulta{Fakulta strojního inženýrství}{Faculty of Mechanical Engineering}
%\fakulta{Fakulta výtvarných umění}{Faculty of Fine Arts}

%% Označení ústavu
% První parametr je název ústavu v originálním jazyce,
% druhý parametr je překlad v angličtině nebo češtině
%\ustav{Ústav automatizace a měřicí techniky}{Department of Control and Instrumentation}
%\ustav{Ústav biomedicínského inženýrství}{Department of Biomedical Engineering}
%\ustav{Ústav elektroenergetiky}{Department of Electrical Power Engineering}
%\ustav{Ústav elektrotechnologie}{Department of Electrical and Electronic Technology}
%\ustav{Ústav fyziky}{Department of Physics}
%\ustav{Ústav jazyků}{Department of Foreign Languages}
%\ustav{Ústav matematiky}{Department of Mathematics}
%\ustav{Ústav mikroelektroniky}{Department of Microelectronics}
%\ustav{Ústav radioelektroniky}{Department of Radio Electronics}
%\ustav{Ústav teoretické a experimentální elektrotechniky}{Department of Theoretical and Experimental Electrical Engineering}
\ustav{Ústav telekomunikací}{Department of Telecommunications}
%\ustav{Ústav výkonové elektrotechniky a elektroniky}{Department of Power Electrical and Electronic Engineering}

\logofakulta[loga/FEKT_zkratka_barevne_PANTONE_CZ]{loga/UTKO_color_PANTONE_CZ}


%% Rok obhajoby
\rok{2018}
\datum{13.\,6.\,2018} % Datum se uplatní pouze v prezentaci k obhajobě

%% Místo obhajoby
% Na titulních stránkách bude automaticky vysázeno VELKÝMI písmeny
\misto{Brno}

%% Abstrakt
\abstrakt{Cílem této práce je rozšíření softwarového generátoru útoků o nové typy útoků,
především o útoky na aplikační vrstvě. Práce obsahuje popis, dělení útoků obecně a
konkrétněji se zaobírá útoky implementovanými. Další částí práce je popis rozřiřované
aplikace, její struktura a způsob rozšíření. Poslední část se zabývá otestováním implementovaných útoků.
}{This work deals with the enhancement of software which generates cyberattacks. These enhancements are focused on application layer of ISO/OSI model. The firsh part of the work contains general description of cyberattacks. Concrete attacks which this work is dealing with
are described more concretely. Next part deals with describing generator software and its enhancement. The last part is describing testing of newly implemented cyberattacks.
}

%% Klíčová slova
\klicovaslova{NTP, SSDP, SNMP, dosgen, útok, trafgen, netsniff-ng, ICMP, DoS, DDoS, Linux, kybernetický útok}%
	{NTP, SSDP, SNMP, dosgen, attack, trafgen, netsniff-ng, ICMP, DoS, DDoS, Linux, cyber attack}

%% Poděkování
\podekovanitext{Rád bych poděkoval vedoucímu bakalářské práce panu doc. Ing.~Janu Hajnému, Ph.D.\ za odborné vedení, konzultace, trpělivost a podnětné návrhy k~práci.}  % do tohoto souboru doplňte údaje o sobě, o názvu práce...

%%%%%%%%%%%%%%%%%%%%%%%%%%%%%%%%%%%%%%%%%%%%%%%%%%%%%%%%%%%%%%%%%%%%%%%%

%%%%%%%%%%%%%%%%%%%%%%%%%%%%%%%%%%%%%%%%%%%%%%%%%%%%%%%%%%%%%%%%%%%%%%%%
%%%%%%     Nastavení polí ve Vlastnostech dokumentu PDF      %%%%%%%%%%%
%%%%%%%%%%%%%%%%%%%%%%%%%%%%%%%%%%%%%%%%%%%%%%%%%%%%%%%%%%%%%%%%%%%%%%%%
%% Při vloženém balíčku 'hyperref' lze použít příkaz '\nastavenipdf'
\nastavenipdf
%  Nastavení polí je možné provést také ručně příkazem:
%\hypersetup{
%  pdftitle={Název studentské práce},    	% Pole 'Document Title'
%  pdfauthor={Autor studenstké práce},   	% Pole 'Author'
%  pdfsubject={Typ práce}, 						  	% Pole 'Subject'
%  pdfkeywords={Klíčová slova}           	% Pole 'Keywords'
%}
%%%%%%%%%%%%%%%%%%%%%%%%%%%%%%%%%%%%%%%%%%%%%%%%%%%%%%%%%%%%%%%%%%%%%%%
\logohlavicka					% vytvoření zkraceného loga VUT FEKT (FEEC) v hlavičce slajdu, nechte odkomentované

\usetheme{VUT} 				% barvy a rozložení prezentace odpovídající VUT FEKT
% alternativne lze pouzit jina berevna temata, napr. 
%\usetheme{Darmstadt} \usecolortheme{default2}
% ale bez zaruky

\begin{document}

% v pripade zakomentovani se zobrazi v pravem dolnim rohu slajdu klikatelne navigacni symboly 
\vypninavigacnisymboly

% snimek s titulni strankou vysazen bez hornich, dolnich a postranich list (volba plain),
% neni tak vysazen ani nadpis snimku
\vytvortitulku

%%%%%%%%%%%%%%%%%%%%%%%%%%%%%%%%%%%%%%%%%%%%%%%%%%%%%%%%%%%%%%%%%%%%%%%

% 1.snimek s cili (zadanim) prace
\begin{frame} 
	% nadpis snímku
	\frametitle{Cíle práce}
	\begin{itemize}
			\item Nastudovat kybernetické útoky
			\begin{itemize}
				\item posouzení implementace
			\end{itemize}
			\item Popsat
				\begin{itemize}
					\item obecně
					\item implementované
				\end{itemize}
			\item Implementovat
				\begin{itemize}
					\item NTP
					\item SNMP
					\item SSDP
				\end{itemize}
			\item Otestovat
	\end{itemize}
\end{frame}

%%%%%%%%%%%%%

%co je to kyberneticky utok, zbezny popis mnou vybranych utoku
\begin{frame}
	\frametitle{Kybernetický útok}
	\begin{columns}[T] 								% prostředí sloupce s umístěním nahoře
		\begin{column}{0.3\textwidth}		% první sloupec
			\begin{itemize}
				\item Co je to?
				\item DoS/DDoS
					\begin{itemize}
						\item C\&C
					\end{itemize}
				\item Spoofing
				\item Reflekce
				\item Amplifikace
			\end{itemize}
		\end{column}
		\begin{column}{0.7\textwidth}
			\begin{figure}%
				\centering
				\vspace{0.1cm}	              % horizontální mezera
				\includegraphics[width=0.99\columnwidth]{obrazky/ddos_schema.png}
			\end{figure}
		\end{column}
	\end{columns}											% ukončení prostředí sloupce
\end{frame} 


%%%%%%%%%%%%%
\begin{frame}
	\frametitle{NTP Flood}
	
	\begin{columns}[T] 								% prostředí sloupce s umístěním nahoře
		\begin{column}{0.4\textwidth}		% první sloupec
			\begin{itemize}
				\item UDP protokol
				\item Reflekce + amplifikace, až 556.9krát
				\begin{itemize}
					\item IP spoofing
				\end{itemize}
				\item \texttt{monlist}
				\begin{itemize}
					\item monitorování
					\item 600 klientů
				\end{itemize}
				\item mapování ze strany útočníků
				\item situace dnes
				\begin{itemize}
					\item CVE-2013-5211
				\end{itemize}
			\end{itemize}
		\end{column}
		%
		\begin{column}{0.6\textwidth}		% druhý sloupec
			\begin{figure}%	
				\centering
				\vspace{1cm}	              % horizontální mezera
				\includegraphics[width=0.8\columnwidth]{obrazky/ntp_flood_schema.png}
			\end{figure}
		\end{column}
	\end{columns}											% ukončení prostředí sloupce
\end{frame}


%%%%%%%%%%%%%
\begin{frame}
	\frametitle{Násobení velikosti odpovědi oproti požadavku}
	\vspace{0.1cm}
		\begin{table}[ht]
			\centering
			\caption{Amplifikační útoky založené na UDP}
			\label{tab:udp_ampl}
			\begin{tabular}{|l|l|}
				\hline
				\textbf{Protokol}      & \textbf{Faktor zvětšení šířky pásma}    \\ \hline
				NTP                    & 556.9                                   \\ \hline
				CharGen                & 358.8                                   \\ \hline
				DNS                    & do 179                                  \\ \hline
				QOTD                   & 140.3                                   \\ \hline
				Quake Network Protocol & 63.9                                    \\ \hline
				BitTorrent             & 4.0 - 54.3                              \\ \hline
				SSDP                   & 30.8                                    \\ \hline
				Kad                    & 16.3                                    \\ \hline
				SNMPv2                 & 6.3                                     \\ \hline
			\end{tabular}
	\end{table}
\end{frame}

%%%%%%%%%%%%%

\begin{frame}
\frametitle{SNMP Flood}

\begin{columns}[T] 								% prostředí sloupce s umístěním nahoře
	\begin{column}{0.4\textwidth}		% první sloupec
		\begin{itemize}
			\item UDP protokol, SNMPv2c
			\item Reflekce + amplifikace, až 6.3krát
			\item Komunikace \\agent - manažer i obráceně
			\item \texttt{community	string} = \textit{public}
			\item \texttt{GetBulkRequest}
			\item OID 1.3.6.1.2.1.1
		\end{itemize}
	\end{column}
	%
	\begin{column}{0.6\textwidth}		% druhý sloupec
		\begin{figure}%
			\centering
			\vspace{1cm}	              % horizontální mezera
			\includegraphics[width=1\columnwidth]{obrazky/snmp_flood_schema.png}
		\end{figure}
	\end{column}
\end{columns}											% ukončení prostředí sloupce
\end{frame}

%%%%%%%%%%%%%

\begin{frame} 
\frametitle{SSDP Flood}

\begin{columns}[T] 								% prostředí sloupce s umístěním nahoře
	\begin{column}{0.4\textwidth}		% první sloupec
		%			Obrázek znázorňuje modelfsdfsdfsfasf:\\[2ex]
		%
		\begin{itemize}
			\item UDP, 1900
			\item Reflekce + amplifikace, až 30.8krát
			\item Komunikace multicast \texttt{239.255.255.250}
			\item UPnP
			\item \texttt{M-SEARCH}, \texttt{NOTIFY}
			\item HTTP hlavička
			\item Odpověď unicast
		\end{itemize}
	\end{column}
	%
	\begin{column}{0.6\textwidth}		% druhý sloupec
		\begin{figure}%
			\centering
			\vspace{1cm}	              % horizontální mezera
			\includegraphics[width=1\columnwidth]{obrazky/ssdp_flood_schema.png}
		\end{figure}
	\end{column}
\end{columns}											% ukončení prostředí sloupce
\end{frame}

%%%%%%%%%%%%%

\begin{frame}
\frametitle{SSDP Flood}
		\begin{figure}%
			\centering
			\vspace{0.4cm}	              % horizontální mezera
			\includegraphics[width=0.659\columnwidth]{obrazky/ssdp_packets_wireshark.png}
		\end{figure}										% ukončení prostředí sloupce
\end{frame}

%%%%%%%%%%%%%

\begin{frame}
\frametitle{DoSgen}
\begin{itemize}
	\item Implementace v C, \CC
	\item \texttt{trafgen}
	\begin{itemize}
		\item \texttt{netsniff-ng}
		\item zero-copy
		\item více jader
	\end{itemize}
	\item Webové rozhraní
	\begin{itemize}
		\item Node.js
		\item HTTPS, 8888
	\end{itemize}
\end{itemize}
\end{frame}

%%%%%%%%%%%%%

\begin{frame}
\frametitle{DoSgen}
\begin{figure}%
	\centering
	\vspace{0.3cm}	              % horizontální mezera
	\includegraphics[width=0.4\columnwidth]{obrazky/dosgen_halaska_diagrampng.png}
	\caption{Struktura aplikace DoSgen}%
\end{figure}
\end{frame}


%%%%%%%%%%%%%

\begin{frame}
\frametitle{DoSgen webové rozhraní}
\begin{itemize}
	\item Aktualizace závislostí (CVE)
	\item Aktualizace inicializačního skriptu (systemd)
	\item Implementace útoků
\end{itemize}
\begin{figure}%
	\centering
	\vspace{0.25cm}	              % horizontální mezera
	\includegraphics[width=0.62\columnwidth]{obrazky/dosgen-webui.png}
\end{figure}
\end{frame}

%%%%%%%%%%%%%

\begin{frame}
\frametitle{DoSgen nástoroje, dokumentace}
\begin{itemize}
\item Ansible playbooks
\begin{itemize}
	\item Prerekvizity útoku
\end{itemize}
\item Kickstart file
\item man stránka
\begin{itemize}
	\item Markdown → roff
\end{itemize}
\item Pomocné skripty (fake\_ntp\_hosts.py, \dots)
\item Dokumentace v příloze práce
\end{itemize}
\end{frame}

%%%%%%%%%%%%%

\begin{frame}
\frametitle{Testování}
\begin{figure}%
	\centering
	\vspace{0.4cm}	              % horizontální mezera
	\includegraphics[width=0.659\columnwidth]{obrazky/lab_schema.png}
	\begin{table}[ht]
		\centering
		%\caption{Propustnost mezi jednotlivými rozhraními (uvedeno v Mb/s).}
		\label{tab:troughput-lab-interfaces}
		\begin{tabular}{|l|l|l|l|l|}
			\hline
			Rozhraní & eth1 & eth0  & enp1  & ens1  \\ \hline
			eth1     &      & 941   & 525   & 940   \\ \hline
			eth0     & 941  &       & 19200 & 19200 \\ \hline
			enp1     & 525  & 19200 &       & 19200 \\ \hline
			ens1     & 940  & 19200 & 19200 &       \\ \hline
		\end{tabular}
	\end{table}
\end{figure}
\end{frame}

%%%%%%%%%%%%%

\begin{frame}
\frametitle{Testování}
\begin{itemize}
	\item Otestovány všechny útoky
	\item Složitější nastavení infrastruktury
	\item Potřeba většího počtu zařízení
	\item Testování na fyzické infrastruktuře vhodnější
	\begin{itemize}
		\item Potíže s virtuálním switchem
	\end{itemize}
\end{itemize}

\begin{figure}%
	\centering
	\vspace{0.4cm}	              % horizontální mezera
	\includegraphics[width=0.92\columnwidth]{obrazky/grafy/graph_ntp_traffic_5ampl_vs6ampl.png}
\end{figure}
\end{frame}

%%%%%%%%%%%%%

\begin{frame} 
	\frametitle{Závěr}
	\begin{itemize}
		\item Teoretický popis útoků
		\item Popis implementovaných útoků
		\begin{itemize}
			\item Mitigace
		\end{itemize}
		\item Rozšíření nástroje DoSgen
		\item Rozšíření webového rozhraní
		\item Dokumentace, pomocné nástroje
		\item Nastavení, konfigurace testovacího prostředí
		\item Implementováno a otestováno
	\end{itemize}
\end{frame}

%%%%%%%%%%%%%

% podekovani
\begin{frame}[c] 
% bez nadpisu snimku
	\frametitle{\mbox{ }}
	\begin{center}
		{\Huge Děkuji za pozornost!}
	\end{center}
\end{frame}

% otázky oponenta
\frame{
\frametitle{Otázky oponenta}
	\emph{Měl na dosahované výsledky měření vliv spuštěný program Wireshark?}\\[1ex]
	%
	Při získávání výsledků program Wireshark spuštěn nebyl právě proto, abych eliminoval veškeré možné aspekty které by měly negativní vliv na výsledky měření.
	Wireshark, tedy \texttt{libpcap}, při spuštění útoku však vykazoval značné zpoždění při zobrazování zachycených paketů. Je tedy možné, že některé ze statisíců paketů \texttt{libpcap} nezachytil.
	\\[3ex]
		
	\emph{Na jakém rozhraní byly pakety zachytávány při jednotlivých testech?}\\[1ex]
	%
	Při ladění konfigurace paketu nástroje DoSgen byly pakety zachytávány na stanici, jež dané pakety přijímala. Důvodem byla nemožnost zachytávat pakety na rozhraní generátoru, jelikož trafgen pracuje pouze v prostoru jádra OS, kdežto Wireshark v uživatelském prostoru.
	Na této stanici bylo možno zachytit jak paket odeslaný generátorem, tak případnou odpověď oběti.
}

\end{document}
