% Soubory musí být v kódování, které je nastaveno v příkazu \usepackage[...]{inputenc}

\documentclass[%
%  draft,    				  % Testovací překlad
  12pt,       				% Velikost základního písma je 12 bodů
  a4paper,    				% Formát papíru je A4
%  oneside,      			% Jednostranný tisk (výchozí)
%% Z následujicich voleb lze použít maximálně jednu:
%	dvipdfm  						% výstup bude zpracován programem 'dvipdfm' do PDF
%	dvips	  						% výstup bude zpracován programem 'dvips' do PS
%	pdftex							% překlad bude proveden programem 'pdftex' do PDF (výchozí)
	unicode,						% Záložky a informace budou v kódování unicode
%% Z následujících voleb lze použít jen jednu:
%english,            % originální jazyk je angličtina
czech              % originální jazyk je čeština (výchozí)
%slovak,             % originální jazyk je slovenčina
]{report}				    	% Dokument třídy 'zpráva'

\usepackage[utf8]		%	Kódování zdrojových souborů je v UTF-8
	{inputenc}					% Balíček pro nastavení kódování zdrojových souborů

\usepackage{graphicx} % Balíček 'graphicx' pro vkládání obrázků
											% Nutné pro vložení log školy a fakulty

\usepackage[
	nohyperlinks				% Nebudou tvořeny hypertextové odkazy do seznamu zkratek
]{acronym}						% Balíček 'acronym' pro sazby zkratek a symbolů
											% Nutné pro použití prostředí 'seznamzkratek' balíčku 'thesis'

\usepackage[
	breaklinks=true,		% Hypertextové odkazy mohou obsahovat zalomení řádku
	hypertexnames=false % Názvy hypertextových odkazů budou tvořeny
											% nezávisle na názvech TeXu
]{hyperref}						% Balíček 'hyperref' pro sazbu hypertextových odkazů
											% Nutné pro použití příkazu 'nastavenipdf' balíčku 'thesis'

\usepackage{pdfpages} % Balíček umožňující vkládat stránky z PDF souborů
                      % Nutné při vkládání titulních listů a zadání přímo
                      % ve formátu PDF z informačního systému

\usepackage{enumitem} % Balíček pro nastavení mezerování v odrážkách
  \setlist{topsep=0pt,partopsep=0pt,noitemsep}

\usepackage{cmap} 		% Balíček cmap zajišťuje, že PDF vytvořené `pdflatexem' je
											% plně "prohledávatelné" a "kopírovatelné"

\usepackage{upgreek}	% Balíček pro sazbu stojatých řeckých písmem
											% např. stojaté pí: \uppi
											% např. stojaté mí: \upmu (použitelné třeba v mikrometrech)
											% pozor, grafická nekompatibilita s fonty typu Computer Modern!

\usepackage{dirtree}		% sazba adresářové struktury

\usepackage[formats]{listings}	% Balíček pro sazbu zdrojových textů
\lstset{
%	Definice jazyka použitého ve výpisech
%    language=[LaTeX]{TeX},	% LaTeX
%	language={Matlab},		% Matlab
	language={C},           % jazyk C
    basicstyle=\ttfamily,	% definice základního stylu písma
    tabsize=2,			% definice velikosti tabulátoru
    inputencoding=utf8,         % pro soubory uložené v kódování UTF-8
    %inputencoding=cp1250,      % pro soubory uložené ve standardním kódování Windows CP1250
		columns=fixed,  %flexible,
		fontadjust=true %licovani sloupcu
    extendedchars=true,
    literate=%  definice symbolů s diakritikou
    {á}{{\'a}}1
    {č}{{\v{c}}}1
    {ď}{{\v{d}}}1
    {é}{{\'e}}1
    {ě}{{\v{e}}}1
    {í}{{\'i}}1
    {ň}{{\v{n}}}1
    {ó}{{\'o}}1
    {ř}{{\v{r}}}1
    {š}{{\v{s}}}1
    {ť}{{\v{t}}}1
    {ú}{{\'u}}1
    {ů}{{\r{u}}}1
    {ý}{{\'y}}1
    {ž}{{\v{z}}}1
    {Á}{{\'A}}1
    {Č}{{\v{C}}}1
    {Ď}{{\v{D}}}1
    {É}{{\'E}}1
    {Ě}{{\v{E}}}1
    {Í}{{\'I}}1
    {Ň}{{\v{N}}}1
    {Ó}{{\'O}}1
    {Ř}{{\v{R}}}1
    {Š}{{\v{S}}}1
    {Ť}{{\v{T}}}1
    {Ú}{{\'U}}1
    {Ů}{{\r{U}}}1
    {Ý}{{\'Y}}1
    {Ž}{{\v{Z}}}1
}

%% Nastavení českého jazyka při sazbě v češtině.
% Pro sazbu češtiny je možné použít mezinárodní balíček 'babel', jenž
% použití doporučujeme pro nové instalace (MikTeX2.8,TeXLive2009), nebo
% národní balíček 'czech', který doporučujeme ve starších instalacích.
% Balíček 'babel' bude správně fungovat pouze ve spojení s programy
% 'latex', 'pdflatex', zatímco balíček 'czech' bude fungovat ve spojení
% s programy 'cslatex', 'pdfcslatex'.
% Varianta A:
\usepackage    				
  {babel}             % Balíček pro sazbu různojazyčných dokumentů; kompilovat (pdf)latexem!
  										% převezme si z parametrů třídy správný jazyk
\usepackage{lmodern}	% vektorové fonty Latin Modern, nástupce půvoních Knuthových Computern Modern fontů
\usepackage{textcomp} % Dodatečné symboly
\usepackage[T1]{fontenc}  % Kódování fontu - mj. kvůli správným vzorům pro dělení slov
% Varianta B:
%\usepackage{czech}   % Alternativní balíček pro sazbu v českém jazyce, kompilovat (pdf)cslatexem!

\usepackage[%
%% Z následujících voleb lze použít pouze jednu
% left,               % Rovnice a popisky plovoucich objektů budou %zarovnány vlevo
  center,             % Rovnice a popisky plovoucich objektů budou zarovnány na střed (vychozi)
%% Z následujících voleb lze použít pouze jednu
semestral						%	sazba zprávy semestrálního projektu
%bachelor						%	sazba bakalářské práce
%diploma						 % sazba diplomové práce
%treatise            % sazba pojednání o dizertační práci
%phd                 % sazba dizertační práce
]{thesis}             % Balíček pro sazbu studentských prací
                      % Musí být vložen až jako poslední, aby
                      % ostatní balíčky nepřepisovaly jeho příkazy

%%%%%%%%%%%%%%%%%%%%%%%%%%%%%%%%%%%%%%%%%%%%%%%%%%%%%%%%%%%%%%%%%
%%%%%%      Definice informací o dokumentu             %%%%%%%%%%
%%%%%%%%%%%%%%%%%%%%%%%%%%%%%%%%%%%%%%%%%%%%%%%%%%%%%%%%%%%%%%%%%

%% Název práce:
%  První parametr je název v originálním jazyce,
%  druhý je překlad v angličtině nebo češtině (pokud je originální jazyk angličtina)
\nazev{Generátor kybernetických útoků}{Cyberattack generator}

%% Jméno a příjmení autora ve tvaru
%  [tituly před jménem]{Křestní}{Příjmení}[tituly za jménem]
\autor{Ondřej}{Gajdušek}

%% Jméno a příjmení vedoucího/školitele včetně titulů
%  [tituly před jménem]{Křestní}{Příjmení}[tituly za jménem]
% Pokud osoba nemá titul za jménem, smažte celý řetězec '[...]'
\vedouci[doc.\ Ing.]{Jan}{Hajný}[Ph.D.]

%% Jméno a příjmení oponenta včetně titulů
%  [tituly před jménem]{Křestní}{Příjmení}[tituly za jménem]
% Pokud nemá titul za jménem, smažte celý řetězec '[...]'
% Uplatní se pouze v prezentaci k obhajobě;
% v případě, že nechcete, aby se na titulním snímku prezentace zobrazoval oponent, pouze jej zakomentujte;
% u obhajoby semestrální práce se oponent nezobrazuje
\oponent[doc.\ Mgr.]{Křestní}{Příjmení}[Ph.D.]

%% Označení oboru studia
% První parametr je obor v originálním jazyce,
% druhý parametr je překlad v angličtině nebo češtině
\oborstudia{Informační bezpečnost}{Information security}

%% Označení fakulty
% První parametr je název fakulty v originálním jazyce,
% druhý parametr je překlad v angličtině nebo v češtině
%\fakulta{Fakulta architektury}{Faculty of Architecture}
\fakulta{Fakulta elektrotechniky a komunikačních technologií}{Faculty of Electrical Engineering and Communication}
%\fakulta{Fakulta chemická}{Faculty of Chemistry}
%\fakulta{Fakulta informačních technologií}{Faculty of Information Technology}
%\fakulta{Fakulta podnikatelská}{Faculty of Business and Management}
%\fakulta{Fakulta stavební}{Faculty of Civil Engineering}
%\fakulta{Fakulta strojního inženýrství}{Faculty of Mechanical Engineering}
%\fakulta{Fakulta výtvarných umění}{Faculty of Fine Arts}

%% Označení ústavu
% První parametr je název ústavu v originálním jazyce,
% druhý parametr je překlad v angličtině nebo češtině
%\ustav{Ústav automatizace a měřicí techniky}{Department of Control and Instrumentation}
%\ustav{Ústav biomedicínského inženýrství}{Department of Biomedical Engineering}
%\ustav{Ústav elektroenergetiky}{Department of Electrical Power Engineering}
%\ustav{Ústav elektrotechnologie}{Department of Electrical and Electronic Technology}
%\ustav{Ústav fyziky}{Department of Physics}
%\ustav{Ústav jazyků}{Department of Foreign Languages}
%\ustav{Ústav matematiky}{Department of Mathematics}
%\ustav{Ústav mikroelektroniky}{Department of Microelectronics}
%\ustav{Ústav radioelektroniky}{Department of Radio Electronics}
%\ustav{Ústav teoretické a experimentální elektrotechniky}{Department of Theoretical and Experimental Electrical Engineering}
\ustav{Ústav telekomunikací}{Department of Telecommunications}
%\ustav{Ústav výkonové elektrotechniky a elektroniky}{Department of Power Electrical and Electronic Engineering}

\logofakulta[loga/FEKT_zkratka_barevne_PANTONE_CZ]{loga/UTKO_color_PANTONE_CZ}


%% Rok obhajoby
\rok{2017}
\datum{1.\,1.\,1970} % Datum se uplatní pouze v prezentaci k obhajobě

%% Místo obhajoby
% Na titulních stránkách bude automaticky vysázeno VELKÝMI písmeny
\misto{Brno}

%% Abstrakt
\abstrakt{Abstrakt práce v~originálním jazyce
}{Překlad abstraktu v~angličtině (nebo češtině pokud je originální jazyk angličtina)
}

%% Klíčová slova
\klicovaslova{Klíčová slova v~originálním jazyce}%
	{Překlad klíčových slov v~angličtině nebo češtině}

%% Poděkování
\podekovanitext{Rád bych poděkoval vedoucímu diplomové práce panu doc. Ing.~Janu Hajnému, Ph.D.\ za odborné vedení, konzultace, trpělivost a podnětné návrhy k~práci.}  % do tohoto souboru doplňte údaje o sobě, o názvu práce...

%%%%%%%%%%%%%%%%%%%%%%%%%%%%%%%%%%%%%%%%%%%%%%%%%%%%%%%%%%%%%%%%%%%%%%%%

%%%%%%%%%%%%%%%%%%%%%%%%%%%%%%%%%%%%%%%%%%%%%%%%%%%%%%%%%%%%%%%%%%%%%%%%
%%%%%%     Nastavení polí ve Vlastnostech dokumentu PDF      %%%%%%%%%%%
%%%%%%%%%%%%%%%%%%%%%%%%%%%%%%%%%%%%%%%%%%%%%%%%%%%%%%%%%%%%%%%%%%%%%%%%
%% Při vloženém balíčku 'hyperref' lze použít příkaz '\nastavenipdf'
\nastavenipdf
%  Nastavení polí je možné provést také ručně příkazem:
%\hypersetup{
%  pdftitle={Název studentské práce},    	% Pole 'Document Title'
%  pdfauthor={Autor studenstké práce},   	% Pole 'Author'
%  pdfsubject={Typ práce}, 						  	% Pole 'Subject'
%  pdfkeywords={Klíčová slova}           	% Pole 'Keywords'
%}
%%%%%%%%%%%%%%%%%%%%%%%%%%%%%%%%%%%%%%%%%%%%%%%%%%%%%%%%%%%%%%%%%%%%%%%

%%%%%%%%%%%%%%%%%%%%%%%%%%%%%%%%%%%%%%%%%%%%%%%%%%%%%%%%%%%%%%%%%%%%%%%
%%%%%%%%%%%       Začátek dokumentu               %%%%%%%%%%%%%%%%%%%%%
%%%%%%%%%%%%%%%%%%%%%%%%%%%%%%%%%%%%%%%%%%%%%%%%%%%%%%%%%%%%%%%%%%%%%%%
\begin{document}


%% Vložení desek generovaných informačním systémem
\includepdf[pages=1,offset=15.4mm -1in]%
  {pdf/student-desky}% název souboru nesmí obsahovat mezery!
% nebo vytvoření desek z balíčku
%\vytvorobalku
\setcounter{page}{1} %resetovani citace stranek - desky se necisluji

%% Vložení titulního listu generovaného informačním systémem
\includepdf[pages=1,offset=15.4mm -1in]%
  {pdf/student-titulka}% název souboru nesmí obsahovat mezery!
% nebo vytvoření titulní stránky z balíčku
%\vytvortitulku
   
%% Vložení zadání generovaného informačním systémem
\includepdf[pages=1,offset=15.4mm -1in]%
  {pdf/student-zadani}% název souboru nesmí obsahovat mezery!
% nebo lze vytvořit prázdný list příkazem ze šablony
%\stranka{}%
%	{\sffamily\Huge\centering ZDE VLOŽIT LIST ZADÁNÍ}%
%	{\sffamily\centering Z~důvodu správného číslování stránek}

%% Vysázení stránky s abstraktem
\vytvorabstrakt

%% Vysázení prohlaseni o samostatnosti
\vytvorprohlaseni

%% Vysázení poděkování
\vytvorpodekovani

%% Vysázení poděkování projektu SIX
% ----------- zakomentujte pokud neodpovida realite
\vytvorpodekovaniSIX

%% Vysázení obsahu
\obsah

%% Vysázení seznamu obrázků
\seznamobrazku

%% Vysázení seznamu tabulek
\seznamtabulek

%% Vysázení seznamu výpisů
\lstlistoflistings

%% Vložení souboru 'text/uvod.tex' s úvodem
% !TeX spellcheck = cs_CZ
\chapter*{Úvod}
\phantomsection
\addcontentsline{toc}{chapter}{Úvod}


V~dnešní době, kdy existují jisté iniciativy, které chtějí prosadit internet jako právo, je internet 
pro moderního člověka, jeho denní činnosti, i firmy prakticky neodmyslitelný \cite{pirati_internet}. 
Dotčeny jsou nejen oblasti podnikání ale také oblasti vzdělání, umění a další. Výpadek této služby má 
mnohdy nepředstavitelné následky jako finanční ztrátu nebo mnohem citelnější jako nefunkčnost eshopů, 
bankovních či jiných informačních systémů. Za těmito výpadky můžeme nejčastěji hledat kybernetický útok, 
což je promyšlené škodlivé jednání útočníků zaměřené na  IT  infrastrukturu  za  účelem  způsobit 
poškození  a  získat  citlivé  či  strategicky  důležité 
informace promyšlené nebo naplnit určitý sociální, ideologický, náboženský, politický nebo jiný záměr 
\cite{Jirasek2012}. Čím více je fungování společnosti na internetu závislejší, tím více jsou tyto hrozby 
závažnější a je třeba se jimi zabývat. Nejčastějším typem útoku je \zkratka{zk_dos}, proti kterému 
v~mnoha případech neexistuje  žádná spolehlivá obrana \cite{akamai_q2_2017}. Útoky tohoto typu mají 
za cíl znepřístupnit danou službu běžící na serveru nebo server samotný, pro běžného uživatele se tak 
tváří býti nedostupný. Útočníkem se může stát kdokoli, ať už je to běžný uživatel, síťový expert nebo 
v~také organizovaná skupina útočníků. Právě tyto skupiny jsou schopny docílit většího efektu a jsou 
schopny napáchat větší škody. Cíl útoku nemusí nutně být národního či nadnárodního rozměru.

Je zjevné, že kybernetickou bezpečnost nelze podcenit a bezesporu zaujímá neméně důležitou roli. Tato 
práce se zabývá popisem \zkratkatext{zk_dos} útoků a rozšířením softwarového generátoru kybernetických útoků 
o~nové typy útoků.

První část popisuje \zk{zk_dos} útoky, přiblížím jejich původ, fungování. Implementované typy útoků 
popíši detailněji.

%% Vložení souboru 'text/reseni' s popisem reseni práce
\chapter[Teoretická část studentské práce]{Teoretická část studentské\\ práce}

Teoretické zázemí studentské práce vhodně rozdělené do částí.

(Struktura navržená v~této šabloně je nejhrubší možná, po konzultaci s~vedoucím je vhodné zvolit přiléhavější.)


%% Vložení souboru 'text/zaver' se závěrem
% !TeX spellcheck = cs_CZ
\chapter{Závěr}
V této práci se zabývám rozšířením generátoru kybernetických útoků o nové útoky. V úvodu této práce jsou definovány kybernetické útoky obecně, útoky, které následně implementuji, jsou popsány detailněji. Práce detailně popisuje typy kybernetických útoků, jejich různé formy a dělení. Je vysvětlen cíl kybernetických útoků, kdo za nimi stojí a jakým způsobem jsou vykonávany.

V druhé části této se zabývám samotným popisem rozšíření nástroje DoSgen, popisuji jeho strukturu a možnosti použití pro mnou zvolené útoky. Dále se zabývám popisem implementace konkrétních útoků a jejich použitím.

V poslední kapitole se věnuji testování nově implementovaných útoků. Jsou přiloženy snímky obrazovky z průběhu testování.

Ve výsledku se mi podařilo implementovat a řádně otestovat pouze jeden útok, jelikož jsem neodhadl komplexnost této práce.

Dále se plánuji zabývat implementaci zbylých nedokončených útoků, konkrétně SNMP a SSDP. Jedním z dalších mých cílů je restrukturalizace kódu, aktualizace nástroje trafgen implementovaného v DoSgenu a vytvoření manuálové stránky pro DoSgen. V neposlední řadě plánuji vytvořit RPM balíček pro Red Hat Linuxové distribuce.

%% Vložení souboru 'text/literatura' se seznamem literatury
% Pro sazbu seznamu literatury použijte jednu z následujících možností

%%%%%%%%%%%%%%%%%%%%%%%%%%%%%%%%%%%%%%%%%%%%%%%%%%%%%%%%%%%%%%%%%%%%%%%%%
%1) Seznam citací definovaný přímo pomocí prostředí literatura / thebibliography

\begin{literatura}{99}

\bibitem{pirati_internet}
Parlament 2017 - Dlouhodobý program - Internet. \textit{Pirátská strana} [online]. [cit. 2017-11-19]. Dostupné z: https://www.pirati.cz/program/dlouhodoby/internet/

\bibitem{Jirasek2012}
JIRÁSEK, Petr, Luděk NOVÁK a Josef POŽÁR. \textit{Výkladový slovník kybernetické bezpečnosti}. Praha: Policejní akademie ČR v Praze, 2012. ISBN 978-80-7251-378-9.

\bibitem{akamai_q2_2017}
State of the Internet / security Q2 2017 Report. In: ARTEAGA, Jose, Dave LEWIS, Chad SEAMAN, et al. \textit{State of the Internet: Quarterly Security Reports} [online]. 2017, s.~2 [cit. 2017-11-21]. Dostupné z: https://www.akamai.com/uk/en/multimedia/documents/state-of-the-internet/q2-2017-state-of-the-internet-security-report.pdf


\end{literatura}


%%%%%%%%%%%%%%%%%%%%%%%%%%%%%%%%%%%%%%%%%%%%%%%%%%%%%%%%%%%%%%%%%%%%%%%%%
%%2) Seznam citací pomocí BibTeXu
%% Při použití je nutné v TeXnicCenter ve výstupním profilu aktivovat spouštění BibTeXu po překladu.
%% Definice stylu seznamu
%\bibliographystyle{unsrturl}
%% Pro českou sazbu lze použít styl czechiso.bst ze stránek
%% http://www.fit.vutbr.cz/~martinek/latex/czechiso.tar.gz
%%\bibliographystyle{czechiso}
%% Vložení souboru se seznamem citací
%\bibliography{text/literatura}
%
%% Následující příkaz je pouze pro ukázku sazby literatury při použití BibTeXu.
%% Způsobí citaci všech zdrojů v souboru odkazy.bib, i když nejsou citovány v textu.
%\nocite{*}

%% Vložení souboru 'text/zkratky' se seznam použitých symbolů, veličin a zkratek
\begin{seznamzkratek}{Distributed Reflection Denial of Service}

	\novazkratka{zk_dos}
		{DoS}
		{Denial of Service}

	\novazkratka{zk_ddos}
		{DDoS}
		{Distributed Denial of Service}

	\novazkratka{zk_drdos}
		{DRDoS}
		{Distributed Reflection Denial of Service}

	\novazkratka{zk_ddosaas}
		{DDoSaaS}
		{Denial of Service as Service}

	\novazkratka{zk_cnc}
		{C\&C} %mezera ok?
		{Command and Control}

	\novazkratka{zk_irc}
		{IRC}
		{Internet Relay Chat}

	\novazkratka{zk_im}
		{IM}
		{Instant messaging}

	\novazkratka{zk_ntp}
		{NTP}
		{Network Time Protocol}

	\novazkratka{zk_http}
		{HTTP}
		{Hypertext Transfer Protocol}

	\novazkratka{zk_tcp}
		{TCP}
		{Transmission Control Protocol}

	\novazkratka{zk_udp}
		{UDP}
		{User Datagram Protocol}

	\novazkratka{zk_ip}
		{IP}
		{Internet Protocol}

	\novazkratka{zk_icmp}
		{ICMP}
		{Internet Control Message Protocol}

	\novazkratka{zk_smtp}
		{SMTP}
		{Simple Mail Transfer Protocol}
		
	\novazkratka{zk_snmp}
		{SNMP}
		{Simple Network Management Protocol}
	
	\novazkratka{zk_cve}
		{CVE}
		{Common Vulnerabilities and Exposures}
\end{seznamzkratek}


%% Začátek příloh
\prilohy

%% Vysázení seznamu příloh
\seznampriloh

%% Vložení souboru 'text/prilohy' s přílohami
\chapter{Dokumentace nástroje DoSgen}
%makefiles zminit

\section{Instalace}
V této kapitole bude popsána instalace nástroje DoSgen na operačním systému Fedora. Jsou zde zmíněny knihovny, jež je třeba instalovat pro jeho plnou funkčnost. 

\subsection{Instalace závislostí}
\noindent Instalace následujících knihoven je nutností, jelikož jsou přímými závislostmi nástroje trafgen. Výjimkou je knihovna \texttt{libssh}, na které je závislý slovníkový útok.
\begin{lstlisting}[language=bash]
dnf -y install flex bison libnl3-devel libssh-devel
\end{lstlisting}

\noindent Dále je třeba nainstalovat nástroje potřebné ke kompilaci kódu.
\begin{lstlisting}[language=bash]
dnf -y install make pkg-config gcc-c++
\end{lstlisting}

\noindent Chceme-li sestavit dokumentaci k nástroji DoSgen, je nutné mít nainstalován balíček \texttt{rubygem-ronn}, jež zajišťuje konverzi manuálové stránky z jednoduše formátovaného souboru v jazyce Markdown do formátu \texttt{roff}, který je používán pro psaní manuálových stránek.

\begin{lstlisting}[language=bash]
dnf -y install rubygem-ronn
\end{lstlisting}

\noindent Webové rozhraní je závislé na \texttt{nodejs} a \texttt{npm} balíčcích.
\begin{lstlisting}[language=bash]
dnf -y install nodejs npm
\end{lstlisting}



\noindent Instalace následujících balíčků je pouze volitelné, jde o užitečné nástroje, které najdou své místo vedle nástroje DoSgen.
\begin{lstlisting}[language=bash]
dnf -y install tcptrack nmap httping john iftop nload \
ntp net-snmp-utils iputils net-tools
\end{lstlisting}

\subsection{Kompilace}
\noindent Dále je potřeba přeložit zdrojový kód. V rámci této práce byly vyvinuty soubory \texttt{Makefile} v každém podadresáři kořenového adresáře. V kořenovém adresáři se nachází hlavní \texttt{Makefile}, který spouští všechny Makefile obsažené v podadresářích. Tyto soubory obsahují příkazy pro kompilaci a instalaci kódu a zajišťují správné pořadí jejich vykonání.

\begin{lstlisting}[language=bash]
make
\end{lstlisting}

V rámci této práce byl vytvořen také Dockerfile, pomocí kterého lze sestavit kontejner s nainstalovaným nástrojem DoSgen. Výhoda tohoto kontejneru spočívá v tom, že procesy spuštěné v kontejneru jsou oddělené od procesů v hostitelské stanici. Tento kontejner staví na operačním systému Fedora ve verzi 27. Sestavení kontejneru a jeho spuštění lze docílit následovně.
\begin{lstlisting}
docker build -t <NÁZEV_OBRAZU_KONTEJNERU> .
docker run <NÁZEV_OBRAZU_KONTEJNERU>
\end{lstlisting}

\subsection{Webové rozhraní}
Pro spuštění webového rozhraní je nutno nainstalovat Node.js balíčky.
\begin{lstlisting}[language=bash]
cd dosgen-web
make install-npm
# nebo
npm install
\end{lstlisting}

Chceme-li spouštět webovou aplikaci jako démona běžícího na pozadí, je nutno nainstalovat inicializační skript \texttt{dosgen-web.init} do adresáře \texttt{/etc/init.d} a enviromentální soubor do \texttt{/etc/sysconfig}. Používáme-li systém se systemd, pak namísto inicializačního skriptu instalujeme jednotku \texttt{dosgen-web.service} do složky \texttt{/usr/lib/systemd/system}.

Tyto kroky není nutné provádět ručně. Pro tento účel je v Makefile patřičný cíl, který provede instalaci těchto souborů.
\begin{lstlisting}[language=bash]
make install
\end{lstlisting}

V souboru s proměnnými prostředí a souboru \texttt{dosgen.js} je nutno nastavit cestu k nástroji DoSgen a jeho binárnímu souboru. V těchto souborech jsou přítomny komentáře, které uživatele navedou.

\subsection{Dokumentace}
Chceme-li si zobrazit manuálovou stránku pro nástroj DoSgen, je třeba provést překlad.
\begin{lstlisting}[language=bash]
cd man
make
\end{lstlisting}

\subsection{RPM balíček}
Kořenový adresář projektu obsahuje soubor \texttt{dosgen.spec}, jež obsahuje kompletní instrukce pro sestavení RPM balíčku pro Linuxové distribuce založené na balíčkovacím systému RPM. Specfile používá příkaz \texttt{make} ke kompilaci zdrojového kódu i jeho následnou instalaci. Pro sestavení balíčku je nutno vytvořit SRPM balíček a poté spustit proces sestavení samotného RPM balíčku. Pro vytvoření výsledného balíčku je možno použít software \texttt{mock}. 

\begin{lstlisting}[language=bash]
make test-srpm
mock -r fedora-rawhide-x86_64 \
rpmbuild/SRPMS/dosgen-*.src.rpm
\end{lstlisting}

\section{Webové rozhraní}
Webové rozhraní lze spustit jako službu na pozadí,

\begin{lstlisting}[language=bash]
service dosgen-web start
\end{lstlisting}

\noindent nebo lze jej spustit přímo z adresáře \texttt{dosgen-web} spuštěním inicializačního skriptu.

\begin{lstlisting}[language=bash]
./dosgen-web.init start
\end{lstlisting}
%todo přidat obrázek webui
Webové rozhraní je dostupné na adrese:
\begin{lstlisting}
https://<IP adresa stanice>:8888
\end{lstlisting}

Po načtení webového rozhraní je nutno přihlásit se údaji zadanými v souboru \texttt{config.json}. Na hlavní stránce v panelu \texttt{Common Settings} je nutno zadat odchozí síťové rozhraní. Čas útoku a počet procesů jsou volitelné. V nabídce \texttt{Add your attack} si lze vybrat požadovaný útok a následně je uživatel vyzván k zadání parametrů pro tento útok v dalším panelu.

Tlačítka \texttt{Run!} a \texttt{Stop!} slouží ke spuštění a zastavení konkrétního útoku.

%TODO ?? \section{Konfigurační jazyk nástroje Trafgen}





\chapter{Obsah přiloženého CD}
% TODO: vlozit pdf
{\small
%
\dirtree{%.
.1 /\DTcomment{kořenový adresář přiloženého CD}.
.2 configure.
.2 Dockerfile.
.2 dosgen.
.3 argsParse.c.
.3 argsParse.cpp.
.3 argsParse.h.
.3 dosgen.c.
.3 httpFlood.cpp.
.3 httpFlood.h.
.3 libdos.h.
.3 Makefile.
.3 packet.
.3 rudy.cpp.
.3 rudy.h.
.3 slowloris.cpp.
.3 slowloris.h.
.3 slowRead.cpp.
.3 slowRead.h.
.3 sshPass.c.
.3 sshPass.h.
.3 trafgen.
.3 trafgen\_configs.h.
.3 trafgenlib.h.
.3 trafgen\_wrapper.c.
.3 trafgen\_wrapper.h.
.2 dosgen-ansible.
.3 ansible.cfg.
.3 hosts.
.3 ntp.yml.
.3 roles.
.4 ntp.
.4 snmp.
.3 snmp.yml.
.2 dosgen.spec.
.2 dosgen-tools.
.2 dosgen-web.
.3 bin.
.4 www.
.3 cert.pem.
.3 config.json.
.3 dosgen-web.env.
.3 dosgen-web.init.
.3 dosgen-web.service.
.3 key.pem.
.3 Makefile.
.3 package.json.
.3 passport.js.
.3 public.
.4 fonts.
.4 images.
.4 javascripts.
.4 stylesheets.
.3 routes.
.4 auth.js.
.4 dosgen.js.
.3 server.js.
.3 views.
.4 dosgen.ejs.
.4 error.ejs.
.4 login.ejs.
.2 machine-provisioning.
.3 ntpserver.
.4 fedora14.
.5 fedora14.ks.
.5 installation\_command.
.2 Makefile.
.2 man.
.3 dosgen.8.ronn.
.3 Makefile.
.3 README.md.
.2 README.md.
}
}

%% Konec dokumentu
\end{document}
